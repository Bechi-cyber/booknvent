% !TEX TS-program = pdflatex
% !TEX encoding = UTF-8 Unicode

\documentclass[12pt, a4paper, oneside]{book}

% Essential packages
\usepackage[utf8]{inputenc}
\usepackage[T1]{fontenc}
\usepackage{lmodern}
\usepackage{microtype}
\usepackage{amsmath, amssymb, amsfonts}
\usepackage{graphicx}
\usepackage[hidelinks]{hyperref}
\usepackage[capitalise]{cleveref}
\usepackage[left=3cm, right=2.5cm, top=2.5cm, bottom=2.5cm]{geometry}
\usepackage{setspace}
\usepackage{booktabs}
\usepackage{caption}
\usepackage{subcaption}
\usepackage{enumitem}
\usepackage{fancyhdr}
\usepackage{titlesec}
\usepackage{xcolor}
\usepackage{listings}
\usepackage{algorithm}
\usepackage{algpseudocode}
\usepackage{tikz}
\usepackage{pgfplots}
\usepackage{float}
\usepackage{csquotes}
\usepackage[backend=biber, style=ieee, sorting=none]{biblatex}

% Bibliography file
\addbibresource{references.bib}

% Custom settings
\onehalfspacing
\setlength{\parindent}{1.5em}
\setlength{\parskip}{0.5em}

% Custom commands
\newcommand{\figref}[1]{Figure~\ref{#1}}
\newcommand{\tabref}[1]{Table~\ref{#1}}
\newcommand{\eqnref}[1]{Equation~\ref{#1}}
\newcommand{\secref}[1]{Section~\ref{#1}}
\newcommand{\chapref}[1]{Chapter~\ref{#1}}
\newcommand{\appref}[1]{Appendix~\ref{#1}}

% Header and footer settings
\pagestyle{fancy}
\fancyhf{}
\fancyhead[LE,RO]{\thepage}
\fancyhead[RE]{\textit{\leftmark}}
\fancyhead[LO]{\textit{\rightmark}}
\renewcommand{\headrulewidth}{0.5pt}
\renewcommand{\footrulewidth}{0pt}

% Title formatting
\titleformat{\chapter}[display]
  {\normalfont\huge\bfseries}{\chaptertitlename\ \thechapter}{20pt}{\Huge}
\titlespacing*{\chapter}{0pt}{50pt}{40pt}

\begin{document}

\chapter{ABSTRACT}

This research presents the design, implementation, and evaluation of LESAVOT, a multimodal steganography platform that integrates text, image, and audio steganographic techniques within a unified, user-friendly system. In an era of increasing digital threats, where cybercrime costs are projected to reach \$10.5 trillion annually by 2025, there is a critical need for advanced data protection strategies that go beyond traditional encryption methods. Steganography, the art of hiding information within innocuous media, offers a complementary security layer by concealing the very existence of sensitive data.

The LESAVOT platform addresses significant gaps in existing steganographic approaches by combining multiple modalities, enhancing embedding capacity, and improving resistance to detection while maintaining a focus on usability. The research employed a mixed-methods approach, including system development, technical evaluation, and user experience assessment with 38 ICT University students from diverse academic backgrounds.

Key findings demonstrate the technical feasibility and practical benefits of a multimodal approach to steganography. The platform achieved high imperceptibility across all modalities, with image steganography using adaptive LSB techniques showing PSNR values above 40 dB, audio steganography maintaining SNR values above 35 dB, and text steganography using zero-width characters proving virtually undetectable without specialized tools. User experience evaluation revealed high usability ratings, with 81.6\% of participants rating the overall experience as "Very Good" or "Excellent," and 84.2\% expressing high confidence in the platform's security features.

The research contributes to the field of information security by demonstrating how multimodal steganography can enhance data protection through increased flexibility, capacity, and security. The LESAVOT platform represents an advancement in both the technical implementation of steganographic techniques and their accessibility to users with varying levels of technical expertise. Recommendations for future work include the development of more advanced adaptive algorithms, cross-modal steganographic techniques, and standardized evaluation metrics for steganographic systems.

\chapter{LIST OF FIGURES AND TABLES}

\section*{Figures}

\begin{itemize}
    \item Figure 1.1: Conceptual Framework of Multimodal Steganography
    \item Figure 2.1: Classification of Steganographic Techniques
    \item Figure 2.2: Comparison of Single-Modal vs. Multimodal Steganography
    \item Figure 3.1: Research Design Framework
    \item Figure 3.2: Sampling Distribution by Academic Program
    \item Figure 4.1: User Satisfaction Ratings for LESAVOT Platform
    \item Figure 4.2: Modality Preference Distribution
    \item Figure 4.3: Use Case Diagram for LESAVOT Platform
    \item Figure 4.4: Activity Diagram for Steganographic Operations
    \item Figure 4.5: Class Diagram of LESAVOT System Architecture
    \item Figure 4.6: Technical Performance Metrics Across Modalities
    \item Figure 4.7: System Implementation Architecture
    \item Figure 4.8: User Interface Screenshots
    \item Figure 4.9: Testing Results Summary
    \item Figure 5.1: Technical Performance Summary
    \item Figure 5.2: User Experience Findings
    \item Figure 5.3: Technical Implications
    \item Figure 5.4: User Experience Implications
    \item Figure 5.5: Practical Applications
    \item Figure 5.6: Recommendations Summary
\end{itemize}

\section*{Tables}

\begin{itemize}
    \item Table 3.1: Participant Demographics by Academic Program
    \item Table 3.2: Research Instruments and Data Collection Methods
    \item Table 4.1: Statistical Analysis of User Feedback
    \item Table 4.2: Comparative Analysis of Steganographic Modalities
    \item Table 4.3: Embedding Capacity Comparison
    \item Table 4.4: Imperceptibility Metrics
    \item Table 4.5: Security Feature Evaluation Results
    \item Table 4.6: Performance Efficiency Metrics
\end{itemize}

\chapter{LIST OF ABBREVIATIONS}

\begin{description}
    \item[App.] Appendix
    \item[Avg.] Average
    \item[Ch.] Chapter
    \item[Cf.] Compare (Latin: confer)
    \item[dB] Decibel
    \item[e.g.] For example (Latin: exempli gratia)
    \item[Ed.] Edition/Editor
    \item[Eq.] Equation
    \item[et al.] And others (Latin: et alii)
    \item[etc.] And so forth (Latin: et cetera)
    \item[Fig.] Figure
    \item[Hz] Hertz
    \item[i.e.] That is (Latin: id est)
    \item[KB] Kilobyte
    \item[MB] Megabyte
    \item[Max.] Maximum
    \item[Min.] Minimum
    \item[No.] Number
    \item[p.] Page
    \item[pp.] Pages
    \item[Prof.] Professor
    \item[Ref.] Reference
    \item[Sec.] Section
    \item[Std.] Standard
    \item[Tab.] Table
    \item[vs.] Versus
\end{description}

\chapter{LIST OF ACRONYMS}

\begin{description}
    \item[AES] Advanced Encryption Standard
    \item[API] Application Programming Interface
    \item[CCTV] Closed-Circuit Television
    \item[CPU] Central Processing Unit
    \item[CSP] Content Security Policy
    \item[CSRF] Cross-Site Request Forgery
    \item[CSS] Cascading Style Sheets
    \item[DCT] Discrete Cosine Transform
    \item[DFT] Discrete Fourier Transform
    \item[DWT] Discrete Wavelet Transform
    \item[FFT] Fast Fourier Transform
    \item[GAN] Generative Adversarial Network
    \item[GUI] Graphical User Interface
    \item[HTML] Hypertext Markup Language
    \item[HTTP] Hypertext Transfer Protocol
    \item[HTTPS] Hypertext Transfer Protocol Secure
    \item[IoT] Internet of Things
    \item[JPEG] Joint Photographic Experts Group
    \item[JS] JavaScript
    \item[JSON] JavaScript Object Notation
    \item[LSB] Least Significant Bit
    \item[MFA] Multi-Factor Authentication
    \item[MP3] MPEG Audio Layer III
    \item[MPEG] Moving Picture Experts Group
    \item[PBKDF2] Password-Based Key Derivation Function 2
    \item[PNG] Portable Network Graphics
    \item[PSNR] Peak Signal-to-Noise Ratio
    \item[PVD] Pixel Value Differencing
    \item[SNR] Signal-to-Noise Ratio
    \item[SQL] Structured Query Language
    \item[SSL] Secure Sockets Layer
    \item[SVG] Scalable Vector Graphics
    \item[TLS] Transport Layer Security
    \item[UI] User Interface
    \item[URL] Uniform Resource Locator
    \item[UX] User Experience
    \item[WAV] Waveform Audio File Format
    \item[XSS] Cross-Site Scripting
\end{description}

\chapter{GLOSSARY OF TERMS}

\begin{description}
    \item[Steganography] The practice of concealing information within other non-suspicious data or a physical object to avoid detection.

    \item[Multimodal Steganography] An advanced steganographic approach that integrates multiple data types (text, image, audio) to enhance security, capacity, and flexibility.

    \item[Steganalysis] The study and practice of detecting hidden information within files that have been concealed using steganography.

    \item[Cover Medium] The carrier file or data (text, image, audio) used to hide secret information in steganography.

    \item[Stego Medium] The resulting file or data after secret information has been embedded into a cover medium.

    \item[Embedding Capacity] The amount of secret data that can be hidden within a cover medium without causing noticeable distortion.

    \item[Imperceptibility] The quality of steganographic techniques that ensures the embedded data cannot be perceived by human senses.

    \item[Robustness] The ability of steganographic methods to withstand various transformations, compressions, or manipulations while preserving the hidden data.

    \item[Least Significant Bit (LSB)] A common technique in image and audio steganography that replaces the least significant bits of the cover medium with bits from the secret message.

    \item[Discrete Cosine Transform (DCT)] A transform used in image steganography that converts spatial domain information to frequency domain, often used in JPEG compression.

    \item[Phase Coding] A technique used in audio steganography that embeds data by modifying the phase values of audio segments.

    \item[Zero-Width Characters] Special Unicode characters with no visible width, used in text steganography to hide information without affecting the visible text.

    \item[Payload] The secret message or data that is embedded within the cover medium.

    \item[Cryptography] The practice of securing communication by transforming information into an unreadable format that can only be decoded with a specific key.

    \item[Encryption] The process of converting information into a code to prevent unauthorized access.

    \item[AES (Advanced Encryption Standard)] A symmetric encryption algorithm widely used to secure sensitive data.

    \item[PSNR (Peak Signal-to-Noise Ratio)] A metric used to measure the quality of images or audio after steganographic embedding.

    \item[SNR (Signal-to-Noise Ratio)] A measure used to compare the level of a desired signal to the level of background noise, often used in audio steganography.

    \item[FFT (Fast Fourier Transform)] An algorithm that converts a signal from its time domain to a representation in the frequency domain, commonly used in audio steganography.

    \item[Adaptive Steganography] Techniques that analyze the characteristics of the cover medium to determine optimal embedding locations or methods.
\end{description}

\chapter{Literature Review: Steganography and Data Hiding Techniques}

\section{Introduction}
In today's digital age, the security of information has become increasingly critical as cyber threats continue to evolve and proliferate at an unprecedented rate. According to Cybersecurity Ventures, global cybercrime costs are projected to reach \$10.5 trillion by 2025, representing a 15\% year-over-year increase since 2021 \cite{CybersecurityVenture2024}. This alarming trend highlights the urgent need for robust, multi-layered security measures that go beyond traditional approaches. Among these measures, steganography stands out as a powerful technique for protecting sensitive information by concealing its very existence, offering a fundamentally different security paradigm compared to encryption alone.

The concept of steganography has ancient roots, dating back to 440 BCE when Herodotus described how messages were tattooed on the shaved heads of slaves, concealed by regrown hair, and carried across enemy lines undetected. Throughout history, various forms of steganography have been employed, from invisible inks used during wartime to microdots utilized during World War II. In the digital era, steganography has evolved dramatically, leveraging the inherent properties of digital media to hide information in ways that are increasingly sophisticated and difficult to detect \cite{JohnsonSmith2023}.

\begin{figure}[htbp]
    \centering
    \includegraphics[width=0.8\textwidth]{figures/historical_steganography.jpg}
    \caption{[......] (......)}
    \label{fig:historical_steganography}
\end{figure}

Unlike cryptography, which makes the content of a message unintelligible but does not hide its existence, steganography aims to conceal the very presence of secret communication. This distinction is crucial in scenarios where the mere suspicion of covert communication could lead to adverse consequences. As Bruce Schneier, a renowned security expert, noted, \enquote{Steganography is not intended to replace cryptography but supplement it.} When combined with strong encryption, steganography creates a powerful security mechanism that addresses both the confidentiality of the message content and the secrecy of its existence.

\begin{figure}[htbp]
    \centering
    \includegraphics[width=0.9\textwidth]{figures/steganography_vs_cryptography.png}
    \caption{[......] (......)}
    \label{fig:stego_vs_crypto}
\end{figure}

The digital revolution has expanded the potential applications of steganography exponentially. The abundance of digital media—images, audio files, videos, and text—provides numerous vehicles for hiding information. Furthermore, the massive volume of digital content exchanged daily creates an ideal environment for steganographic communication to remain undetected. According to recent statistics, over 500 million tweets, 4 million hours of YouTube content, and 95 million Instagram posts are shared daily, creating a vast ocean of potential cover media for steganographic purposes \cite{CybersecurityVenture2024}.

This chapter provides a comprehensive review of steganography and related data hiding techniques, examining both historical contexts and cutting-edge developments. It begins with an overview of steganographic concepts and classifications, followed by an in-depth examination of various steganographic methods across different media types. The chapter also explores cryptographic techniques that complement steganography, reviews recent advancements in the field, and identifies gaps in the existing literature that this research aims to address.

The integration of steganography with modern technologies such as artificial intelligence and machine learning represents a significant frontier in information security. These technologies enable more sophisticated embedding algorithms that can adapt to the characteristics of the cover medium, making the hidden information more resistant to detection. Additionally, they facilitate the development of more effective steganalysis techniques, creating an ongoing arms race between information hiding and detection methods.

The practical applications of steganography extend beyond traditional security domains. In digital watermarking, steganographic techniques are used to embed copyright information or ownership identifiers in digital media. In medical imaging, patient information can be securely embedded within the images themselves, ensuring privacy while maintaining the association between the data and the image. In military and intelligence operations, steganography provides a means for covert communication that can evade surveillance systems designed to detect encrypted messages.

\begin{figure}[htbp]
    \centering
    \includegraphics[width=0.8\textwidth]{figures/steganography_applications.jpg}
    \caption{[......] (......)}
    \label{fig:stego_applications}
\end{figure}

By understanding both traditional approaches and cutting-edge developments in steganography, this research establishes a foundation for the multimodal steganographic system proposed in subsequent chapters. This system aims to leverage the unique characteristics of different media types to create a more robust and versatile information hiding framework that can address the evolving challenges in digital security.

\section{Review of Existing Data Hiding Techniques and Applications}

\subsection{Overview and Classification of Steganographic Techniques}
Steganography, originating from the Greek word \enquote{steganos} meaning \enquote{covered,} and \enquote{graphe} meaning \enquote{writing,} refers to the techniques of concealing secret information within an ordinary file or message in such a way that it cannot be easily detected, unlike cryptography, which only obscures the content of a message but doesn't hide the existence of the message itself \cite{Research2021}. The primary objective is to enable secure communication by ensuring that the hidden message remains undetected by unauthorized entities \cite{JohnsonSmith2023}.

Over the years, various methods have been developed to hide information within digital media, including images, audio, and text. The field of steganography has expanded as cybersecurity concerns grow, especially in environments where privacy is paramount, such as in military communication and digital forensics \cite{Liu2023}. These techniques exploit the redundancy in the host medium, whether in an image's pixels, an audio file's waveform, or the structure of text, to hide secret data in a way that is imperceptible to the human eyes or ears \cite{Smith2022}.

Steganographic techniques can be classified into several categories based on their implementation approaches:

\subsubsection{Technical Steganography}
Technical steganography involves the use of scientific methods and technological tools to hide information within digital media. This category encompasses most modern digital steganography techniques, including those applied to images, audio, and video. Technical steganography relies on the manipulation of digital data structures and formats to embed secret information without perceptible changes to the carrier medium \cite{ZhangLiu2021}.

The fundamental principle behind technical steganography is the exploitation of redundancy in digital media. All digital files contain some degree of redundancy—data that can be modified without significantly affecting the perceived quality of the media. For example, in a 24-bit color image, changing the least significant bit of each color channel (red, green, and blue) results in color changes that are imperceptible to the human eye. Similarly, in audio files, certain frequency components can be modified without audible effects due to the limitations of human hearing.

\begin{figure}[htbp]
    \centering
    \includegraphics[width=0.85\textwidth]{figures/digital_media_redundancy.png}
    \caption{[......] (......)}
    \label{fig:digital_redundancy}
\end{figure}

Technical steganography can be categorized into several approaches based on the domain in which the embedding occurs:

\begin{itemize}[leftmargin=*]
    \item \textbf{Spatial Domain Techniques}: These methods directly modify the raw data of the digital medium. The most common example is Least Significant Bit (LSB) substitution in images, where the least significant bits of pixel values are replaced with bits from the secret message. While simple to implement, these methods can be vulnerable to statistical analysis if not carefully designed. Advanced spatial domain techniques include:
    \begin{itemize}
        \item \textbf{Pixel Value Differencing (PVD)}: This method uses the difference between adjacent pixels to determine how many bits can be embedded in a particular pixel, allowing for variable-length embedding that adapts to the characteristics of the image.
        \item \textbf{Edges-Based Embedding}: This approach targets edge regions in images, where modifications are less noticeable due to the natural variation in pixel values at edges.
        \item \textbf{Palette-Based Methods}: For indexed color images, these techniques modify the color palette or the mapping between pixels and palette entries to encode information.
    \end{itemize}

    \begin{figure}[htbp]
        \centering
        \includegraphics[width=0.75\textwidth]{figures/lsb_steganography.png}
        \caption{[......] (......)}
        \label{fig:lsb_steganography}
    \end{figure}

    \item \textbf{Transform Domain Methods}: These techniques transform the cover medium into a different domain (typically a frequency domain) before embedding the secret data. Common transformations include:
    \begin{itemize}
        \item \textbf{Discrete Cosine Transform (DCT)}: Used in JPEG compression, DCT transforms spatial data into frequency components. Steganographic methods can modify the DCT coefficients to embed data, often targeting the middle-frequency components that are less perceptually significant.
        \item \textbf{Discrete Wavelet Transform (DWT)}: This transformation decomposes the signal into different frequency bands, allowing for more sophisticated embedding that can withstand various attacks and transformations. DWT-based methods often offer better resistance to compression and filtering operations.
        \item \textbf{Discrete Fourier Transform (DFT)}: This transformation represents the signal in terms of its frequency components, which can be modified to carry hidden information. DFT-based methods are particularly useful for audio steganography.
    \end{itemize}

    \begin{figure}[htbp]
        \centering
        \includegraphics[width=0.8\textwidth]{figures/transform_domain_steganography.png}
        \caption{[......] (......)}
        \label{fig:transform_domain}
    \end{figure}

    \item \textbf{Spread Spectrum Techniques}: Borrowed from telecommunications, these methods spread the secret message across a wide frequency band, making it difficult to detect without knowledge of the spreading parameters. The hidden data appears as low-level noise, which is often indistinguishable from the natural noise present in the cover medium. Two main approaches are:
    \begin{itemize}
        \item \textbf{Direct Sequence Spread Spectrum (DSSS)}: The secret message is modulated with a pseudorandom noise sequence before being added to the cover medium.
        \item \textbf{Frequency Hopping Spread Spectrum (FHSS)}: The secret message is transmitted by rapidly switching among different frequency channels according to a pseudorandom sequence.
    \end{itemize}

    \item \textbf{Statistical Methods}: These techniques exploit statistical properties of the cover medium to embed information. They often modify the distribution of certain features in the cover medium to encode the secret message. Examples include:
    \begin{itemize}
        \item \textbf{Model-Based Steganography}: This approach models the statistical properties of the cover medium and modifies it in a way that preserves these properties while embedding the secret message.
        \item \textbf{Histogram-Based Methods}: These techniques modify the histogram of the cover medium (e.g., the distribution of pixel values in an image) to encode information while maintaining the overall statistical characteristics.
        \item \textbf{Quantization-Based Methods}: These approaches quantize certain features of the cover medium according to a predefined rule to embed the secret message.
    \end{itemize}

    \item \textbf{Adaptive Steganography}: These methods analyze the characteristics of the cover medium to determine the optimal embedding locations and parameters. They adapt the embedding process based on factors such as:
    \begin{itemize}
        \item \textbf{Content Complexity}: Regions with higher complexity (e.g., textured areas in images) can typically accommodate more modifications without perceptible changes.
        \item \textbf{Perceptual Models}: These methods incorporate models of human perception (visual or auditory) to identify regions where modifications are less likely to be noticed.
        \item \textbf{Statistical Properties}: The embedding process adapts to maintain the statistical properties of the cover medium, making the stego-object resistant to statistical steganalysis.
    \end{itemize}
\end{itemize}

The effectiveness of technical steganography is evaluated using several key metrics:

\begin{itemize}[leftmargin=*]
    \item \textbf{Imperceptibility}: The degree to which the presence of hidden information is undetectable by human senses. This is often measured using metrics such as Peak Signal-to-Noise Ratio (PSNR), Structural Similarity Index (SSIM), and Just Noticeable Difference (JND).

    \item \textbf{Capacity}: The amount of secret information that can be embedded in the cover medium, typically measured in bits per pixel (bpp) for images or bits per second (bps) for audio. Higher capacity is desirable but often comes at the cost of reduced imperceptibility or robustness.

    \item \textbf{Robustness}: The ability of the hidden information to survive common processing operations such as compression, filtering, or format conversion. Robustness is particularly important for applications such as digital watermarking.

    \item \textbf{Security}: The resistance of the steganographic system to detection by steganalysis techniques. This includes both visual/auditory detection and statistical analysis.

    \item \textbf{Computational Efficiency}: The computational resources required for embedding and extracting the hidden information, which is important for practical applications, especially those involving real-time processing.
\end{itemize}

Recent advances in technical steganography have focused on improving these metrics through the application of machine learning and artificial intelligence. For example, deep learning models have been used to identify optimal embedding locations in images, significantly enhancing the security of the steganographic system while maintaining high capacity and imperceptibility \cite{WuZhang2022}. These approaches represent the cutting edge of technical steganography, pushing the boundaries of what is possible in information hiding.

\begin{figure}[htbp]
    \centering
    \includegraphics[width=0.85\textwidth]{figures/ai_steganography.jpg}
    \caption{[......] (......)}
    \label{fig:ai_steganography}
\end{figure}

\subsubsection{Linguistic Steganography}
Linguistic steganography focuses on hiding information within text by manipulating linguistic properties. This approach leverages the flexibility of natural language to conceal messages without altering the apparent meaning or structure of the text \cite{ShubhamPatel2020}.

Linguistic steganography can be further divided into:
\begin{itemize}[leftmargin=*]
    \item Syntactic methods (manipulating punctuation, word order)
    \item Semantic methods (using synonyms, paraphrasing)
    \item Feature-based methods (altering text formatting, spacing)
\end{itemize}

\begin{figure}[htbp]
    \centering
    \includegraphics[width=0.75\textwidth]{figures/linguistic_steganography.png}
    \caption{[......] (......)}
    \label{fig:linguistic_stego}
\end{figure}

Recent advancements in linguistic steganography have incorporated natural language processing and deep learning techniques to generate steganographic text that maintains high linguistic quality while carrying hidden information \cite{GuptaSharma2022}.

\subsubsection{Semagrams}
Semagrams represent a class of steganographic techniques that use visual symbols or signs to convey hidden messages. Unlike technical or linguistic steganography, semagrams often rely on physical arrangements or visual patterns that can be interpreted by those who understand the encoding system \cite{JohnsonSmith2023}.

Semagrams can be categorized into:
\begin{itemize}[leftmargin=*]
    \item Visual semagrams (using images, drawings, or doodles)
    \item Text semagrams (arranging text to form patterns or shapes)
\end{itemize}

While traditional semagrams were often physical in nature, modern digital implementations have adapted these concepts for use in electronic communications and social media platforms.

\subsubsection{Open Codes}
Open codes refer to steganographic methods where the message appears to be about one topic but contains hidden meaning about another. The hidden message is concealed within the apparent message through prearranged meanings or interpretations known only to the intended recipients \cite{ShubhamPatel2020}.

Examples of open codes include:
\begin{itemize}[leftmargin=*]
    \item Innocent-looking messages with double meanings
    \item Predetermined phrases that signal specific actions or information
    \item Contextual references that have special significance to informed parties
\end{itemize}

Open codes have historical significance in espionage and military communications but continue to be relevant in modern digital communications where surveillance is a concern.

\subsubsection{Jargon Code}
Jargon code involves the use of specialized language or terminology that appears normal to outsiders but conveys hidden meanings to those familiar with the code. This approach relies on shared knowledge between the sender and recipient regarding the interpretation of specific terms or phrases \cite{ShubhamPatel2020}.

Jargon codes can be implemented through:
\begin{itemize}[leftmargin=*]
    \item Specialized vocabulary with alternative meanings
    \item Profession-specific terminology used out of context
    \item Slang or colloquialisms with hidden interpretations
\end{itemize}

In modern applications, jargon codes have evolved to include internet slang, memes, and platform-specific expressions that can carry covert messages while appearing as normal communication to uninformed observers.

\subsubsection{Covered or Concealment Ciphers}
Covered or concealment ciphers combine elements of both steganography and cryptography. In these systems, the message is first encrypted and then hidden within a carrier medium, providing two layers of protection: the encryption obscures the content, while the steganographic technique conceals its existence \cite{Villalba2023}.

\begin{figure}[htbp]
    \centering
    \includegraphics[width=0.8\textwidth]{figures/covered_cipher_diagram.png}
    \caption{Diagram illustrating the process of covered ciphers, showing the combination of encryption and steganographic concealment}
    \label{fig:covered_cipher}
\end{figure}

The dual protection offered by covered ciphers makes them particularly effective for high-security applications where both the content and existence of communications must be protected from unauthorized access.

\subsection{Multimodal Steganography}
Multimodal steganography represents an advanced approach that integrates multiple steganographic techniques across different media types within a unified framework. This emerging field addresses the limitations of single-modality steganography by leveraging the unique characteristics and advantages of various media formats, creating more robust, versatile, and secure information hiding systems \cite{ZhangWu2023}.

\subsubsection{Fundamentals of Multimodal Steganography}
The core concept of multimodal steganography involves distributing secret information across multiple carrier media types, such as text, images, audio, and video. This approach offers several fundamental advantages over traditional single-modality steganography:

\begin{itemize}[leftmargin=*]
    \item \textbf{Enhanced Security}: By distributing the secret message across multiple media types, multimodal steganography creates an additional layer of security. Even if one component is compromised, the complete message remains protected unless all components are successfully attacked.

    \item \textbf{Increased Capacity}: Different media types offer varying embedding capacities. By utilizing multiple carriers, the overall capacity for hiding information increases significantly compared to using a single medium.

    \item \textbf{Improved Robustness}: Each media type has different vulnerabilities to steganalysis and processing operations. A multimodal approach can strategically distribute information to minimize the impact of these vulnerabilities, enhancing overall robustness.

    \item \textbf{Adaptability}: Multimodal systems can adapt to available resources and security requirements by dynamically selecting the most appropriate combination of media types and embedding techniques for a given scenario.
\end{itemize}

\subsubsection{Architectures for Multimodal Steganography}
Multimodal steganographic systems can be designed with various architectural approaches, each offering different trade-offs between security, complexity, and usability:

\begin{itemize}[leftmargin=*]
    \item \textbf{Sequential Architecture}: In this approach, the secret message is processed sequentially through different media types. For example, a message might first be embedded in text, then the resulting stego-text could be embedded in an image, and finally, the stego-image could be embedded in audio. This creates a nested structure of steganographic layers, significantly increasing security but also complexity.

    \item \textbf{Parallel Architecture}: This architecture distributes different portions of the secret message across multiple media types simultaneously. Each medium carries a distinct segment of the message, and all segments must be extracted and combined to reconstruct the complete message. This approach offers good security while maintaining reasonable complexity.

    \item \textbf{Hybrid Architecture}: Combining elements of both sequential and parallel approaches, hybrid architectures offer flexibility in how information is distributed. For example, a system might split a message into segments, apply different encryption methods to each segment, and then embed these encrypted segments across various media types using different steganographic techniques.

    \item \textbf{Adaptive Architecture}: These advanced systems analyze the characteristics of available carrier media and the secret message to dynamically determine the optimal distribution strategy. Machine learning algorithms can be employed to make these decisions based on factors such as media quality, channel conditions, and security requirements.
\end{itemize}

\begin{figure}[htbp]
    \centering
    \begin{tikzpicture}[node distance=2cm, auto]
        % Define styles
        \tikzstyle{block} = [rectangle, draw, fill=blue!20, text width=8em, text centered, rounded corners, minimum height=3em]
        \tikzstyle{line} = [draw, -latex']

        % Place nodes
        \node [block] (secret) {Secret Message};
        \node [block, below of=secret, xshift=-4cm] (text) {Text Steganography};
        \node [block, below of=secret] (image) {Image Steganography};
        \node [block, below of=secret, xshift=4cm] (audio) {Audio Steganography};
        \node [block, below of=text, xshift=2cm, yshift=-1cm] (combine) {Message Reconstruction};

        % Draw edges
        \path [line] (secret) -- (text);
        \path [line] (secret) -- (image);
        \path [line] (secret) -- (audio);
        \path [line] (text) -- (combine);
        \path [line] (image) -- (combine);
        \path [line] (audio) -- (combine);
    \end{tikzpicture}
    \caption{Parallel architecture for multimodal steganography, showing distribution of a secret message across multiple media types}
    \label{fig:multimodal_architecture}
\end{figure}

\subsubsection{Integration Strategies}
Effective multimodal steganography requires careful integration of different techniques to ensure seamless operation and maximum security. Several key integration strategies have emerged in recent research:

\begin{itemize}[leftmargin=*]
    \item \textbf{Message Fragmentation and Reassembly}: This approach involves breaking the secret message into fragments, distributing these fragments across different media types, and implementing a robust protocol for reassembling the fragments during extraction. The fragmentation pattern itself can serve as an additional security layer if kept secret.

    \item \textbf{Cross-Modal Keys}: In this strategy, information embedded in one medium serves as the key for extracting information from another medium. For example, an image might contain the password needed to extract a message hidden in audio, creating an interdependency that enhances security.

    \item \textbf{Complementary Embedding}: This technique leverages the complementary strengths of different media types. For instance, text might be used for embedding small but critical control information, images for larger data chunks requiring visual verification, and audio for time-sensitive or sequential data.

    \item \textbf{Redundant Embedding}: Critical portions of the secret message are embedded redundantly across multiple media types to ensure recoverability even if some carriers are compromised or damaged. Error correction codes can be applied to further enhance robustness.
\end{itemize}

\subsubsection{Security Considerations in Multimodal Steganography}
While multimodal steganography offers enhanced security through its distributed nature, it also introduces unique security considerations that must be addressed:

\begin{itemize}[leftmargin=*]
    \item \textbf{Consistency Across Modalities}: Inconsistencies between different stego-objects might raise suspicion. For example, if an image and accompanying text both contain hidden information but were created using incompatible techniques, statistical analysis might reveal anomalies when the objects are analyzed together.

    \item \textbf{Increased Attack Surface}: Using multiple media types potentially exposes the system to a wider range of steganalysis techniques. Each additional modality introduces its own vulnerabilities that must be mitigated.

    \item \textbf{Complexity Management}: More complex systems generally offer better security but may also introduce more points of failure and usability challenges. Balancing security with practical usability is crucial for real-world applications.

    \item \textbf{Key Management}: Multimodal systems often require multiple keys or passwords for different components. Secure management of these keys becomes a critical consideration to prevent unauthorized access while ensuring legitimate users can retrieve the hidden information.
\end{itemize}

\subsubsection{Applications of Multimodal Steganography}
The versatility of multimodal steganography makes it suitable for a wide range of applications across various domains:

\begin{itemize}[leftmargin=*]
    \item \textbf{Enhanced Secure Communication}: Military, diplomatic, and intelligence agencies can use multimodal steganography for covert communication that resists sophisticated surveillance and steganalysis.

    \item \textbf{Multimedia Authentication}: Multimodal approaches can embed authentication codes across different components of multimedia packages (e.g., text, images, and audio in a news article), creating a robust verification system that ensures integrity across all elements.

    \item \textbf{Digital Rights Management}: Copyright information can be embedded across multiple aspects of digital content, making it more difficult to remove or circumvent compared to single-modality watermarking.

    \item \textbf{Medical Data Security}: Patient information can be distributed across medical records in different formats (text reports, medical images, and audio notes), enhancing privacy while maintaining the association between related data.

    \item \textbf{IoT Security}: In Internet of Things environments, multimodal steganography can secure communication between devices that process different types of media, creating a more robust security framework for sensitive data exchange.
\end{itemize}

\subsubsection{Challenges and Future Directions}
Despite its promising potential, multimodal steganography faces several challenges that present opportunities for future research:

\begin{itemize}[leftmargin=*]
    \item \textbf{Standardization}: The field currently lacks standardized frameworks, protocols, and evaluation metrics specifically designed for multimodal approaches, making it difficult to objectively compare different systems.

    \item \textbf{Computational Overhead}: Managing multiple steganographic techniques across different media types increases computational requirements, potentially limiting applications in resource-constrained environments.

    \item \textbf{Synchronization}: Ensuring proper synchronization between different components of a multimodal system, especially when carriers might be transmitted through different channels or subjected to different processing operations, remains challenging.

    \item \textbf{Usability}: The increased complexity of multimodal systems can create usability barriers for non-expert users. Developing intuitive interfaces and automated processes that hide this complexity is an important area for future work.
\end{itemize}

Future directions in multimodal steganography research include the integration of artificial intelligence for adaptive embedding across modalities, the development of quantum-resistant multimodal techniques, and the creation of standardized evaluation frameworks specifically designed for multimodal systems. As digital communication continues to evolve toward rich multimedia experiences, multimodal steganography is positioned to play an increasingly important role in comprehensive information security strategies \cite{ZhangWu2023}.

\section{The LESAVOT Multimodal Steganography Platform}
This section introduces the LESAVOT platform, a novel implementation of multimodal steganography that integrates text, image, and audio steganographic techniques within a unified web-based framework. The platform represents a practical application of the theoretical concepts discussed in previous sections, addressing several key challenges in multimodal steganography while providing an accessible and user-friendly interface for secure information hiding.

\subsection{Platform Overview and Architecture}
The LESAVOT platform (an acronym for "Less you see, more you hide") is designed as a comprehensive steganographic solution that enables users to conceal sensitive information across multiple media types. The platform's name reflects its core philosophy: "The more you look, the less you see," emphasizing the paradoxical nature of steganography where increased scrutiny does not necessarily lead to better detection of hidden information.

\subsubsection{System Architecture}
LESAVOT employs a modular architecture that separates the user interface, steganographic processing, and security components while maintaining seamless integration between them. The high-level architecture consists of the following key components:

\begin{itemize}[leftmargin=*]
    \item \textbf{User Interface Layer}: A responsive web-based interface that provides access to all steganographic functions across different media types. The interface is designed with both usability and security in mind, guiding users through the process of selecting appropriate media, entering messages, and applying security parameters.

    \item \textbf{Steganographic Processing Modules}: Specialized modules for text, image, and audio steganography, each implementing state-of-the-art techniques for their respective media types. These modules operate independently but share a common API, allowing for consistent interaction with the rest of the system.

    \item \textbf{Security Layer}: Handles encryption, authentication, and key management across all steganographic operations. This layer ensures that even if the steganographic concealment is compromised, the hidden information remains protected through strong cryptographic measures.

    \item \textbf{Storage and History Management}: Maintains records of steganographic operations while implementing secure storage practices to protect sensitive information. This component enables users to track their activities and retrieve previous stego-objects when needed.
\end{itemize}

The platform implements a hybrid architectural approach to multimodal steganography, allowing both parallel embedding (distributing a message across multiple media simultaneously) and sequential processing (applying multiple layers of steganography in sequence) depending on user requirements and security needs.

\begin{figure}[htbp]
    \centering
    \begin{tikzpicture}[node distance=1.5cm, auto]
        % Define styles
        \tikzstyle{block} = [rectangle, draw, fill=blue!10, text width=10em, text centered, rounded corners, minimum height=3em]
        \tikzstyle{line} = [draw, -latex']

        % Place nodes
        \node [block] (ui) {User Interface Layer};
        \node [block, below of=ui, xshift=-5cm] (text) {Text Steganography Module};
        \node [block, below of=ui] (image) {Image Steganography Module};
        \node [block, below of=ui, xshift=5cm] (audio) {Audio Steganography Module};
        \node [block, below of=image, yshift=-1cm] (security) {Security Layer};
        \node [block, below of=security] (storage) {Storage \& History Management};

        % Draw edges
        \path [line] (ui) -- (text);
        \path [line] (ui) -- (image);
        \path [line] (ui) -- (audio);
        \path [line] (text) -- (security);
        \path [line] (image) -- (security);
        \path [line] (audio) -- (security);
        \path [line] (security) -- (storage);
    \end{tikzpicture}
    \caption{LESAVOT platform architecture showing the modular components and their interactions}
    \label{fig:lesavot_architecture}
\end{figure}

\subsubsection{Implementation Technologies}
LESAVOT is implemented as a web application to ensure maximum accessibility across different devices and operating systems. The technology stack includes:

\begin{itemize}[leftmargin=*]
    \item \textbf{Frontend}: HTML5, CSS3, and JavaScript with modern frameworks for responsive design and interactive user experience. The interface employs progressive enhancement techniques to ensure functionality across different browsers and devices.

    \item \textbf{Backend}: Server-side processing for computationally intensive steganographic operations, implemented using secure coding practices and regular security audits to prevent vulnerabilities.

    \item \textbf{Cryptography}: Industry-standard encryption algorithms (AES-256) for securing hidden messages before steganographic embedding, with proper key derivation functions (PBKDF2) to strengthen user-provided passwords.

    \item \textbf{Database}: Secure storage for user accounts and operation history, with appropriate data minimization and protection measures to safeguard sensitive information.
\end{itemize}

\subsection{Steganographic Techniques Implementation}
The LESAVOT platform implements a variety of steganographic techniques across different media types, carefully selected to balance security, capacity, and robustness. Each implementation incorporates both established methods and novel approaches developed specifically for the platform.

\subsubsection{Text Steganography Implementation}
The text steganography module in LESAVOT employs multiple techniques to accommodate different use cases and security requirements:

\begin{itemize}[leftmargin=*]
    \item \textbf{Syntactic Transformation}: Implements reversible syntactic transformations that preserve meaning while encoding binary data. The system analyzes the input text to identify suitable transformation opportunities and applies them selectively to minimize stylistic anomalies.

    \item \textbf{Zero-Width Character Insertion}: Embeds data using invisible Unicode characters (zero-width spaces, joiners, and non-joiners) strategically placed between visible characters. This technique offers high capacity with minimal visual impact but requires careful implementation to ensure compatibility across different text processing systems.

    \item \textbf{Synonym Substitution}: Utilizes a comprehensive synonym database with context awareness to replace words with semantically equivalent alternatives according to the bits of the hidden message. The implementation includes linguistic analysis to ensure natural language flow and maintain the original text's tone and style.
\end{itemize}

The text steganography module also incorporates adaptive selection of techniques based on the characteristics of the cover text and the security requirements specified by the user. For example, formal documents might use more conservative approaches, while casual communications might employ techniques with higher capacity.

\subsubsection{Image Steganography Implementation}
The image steganography module supports multiple image formats (JPEG, PNG, GIF) and implements several embedding techniques:

\begin{itemize}[leftmargin=*]
    \item \textbf{Adaptive LSB Substitution}: Rather than using a fixed LSB approach, the system analyzes image characteristics to determine optimal bit planes and pixel locations for embedding. The algorithm prioritizes textured regions and areas with high color variation where modifications are less perceptible.

    \item \textbf{DCT Domain Embedding}: For JPEG images, the system implements coefficient modification in the Discrete Cosine Transform domain, targeting medium-frequency coefficients to balance robustness against compression with imperceptibility.

    \item \textbf{Edge-Based Embedding}: Identifies edge regions in the image and concentrates embedding in these areas, exploiting the human visual system's reduced sensitivity to changes in high-contrast boundaries.

    \item \textbf{Spread Spectrum Technique}: Implements a spread spectrum approach that distributes the hidden message across the image using a pseudorandom sequence, enhancing security against statistical steganalysis.
\end{itemize}

The module automatically selects the most appropriate technique based on the image type, content characteristics, and user-specified priorities (capacity vs. imperceptibility vs. robustness).

\subsubsection{Audio Steganography Implementation}
The audio steganography module supports common audio formats (WAV, MP3) and implements several techniques:

\begin{itemize}[leftmargin=*]
    \item \textbf{Phase Coding}: Modifies the phase components of audio segments to encode information, exploiting the human auditory system's relative insensitivity to phase changes compared to amplitude variations.

    \item \textbf{Echo Hiding}: Introduces subtle echoes with varying parameters (delay, initial amplitude, decay rate) to encode binary data while maintaining audio quality. The implementation includes psychoacoustic modeling to ensure the echoes remain below the threshold of perception.

    \item \textbf{Spread Spectrum Audio Steganography}: Distributes the hidden message across the frequency spectrum of the audio signal, making it resistant to detection and various signal processing operations.

    \item \textbf{Adaptive Bit Modification}: Analyzes the audio signal's characteristics to identify optimal locations and bit depths for embedding, concentrating modifications in louder segments and frequency bands where human hearing is less sensitive.
\end{itemize}

Like the image module, the audio steganography component employs adaptive technique selection based on audio characteristics, content type (speech, music, ambient sounds), and user requirements.

\subsection{Security Features}
LESAVOT implements multiple layers of security to protect both the steganographic process and the hidden information:

\begin{itemize}[leftmargin=*]
    \item \textbf{Pre-embedding Encryption}: All hidden messages are encrypted using AES-256 before steganographic embedding, ensuring that even if the steganographic layer is compromised, the message content remains protected.

    \item \textbf{Password-Based Protection}: User-provided passwords are processed through PBKDF2 with high iteration counts and salting to derive strong encryption keys, protecting against brute-force and dictionary attacks.

    \item \textbf{Multi-factor Authentication}: The platform implements multi-factor authentication for user accounts, adding an additional layer of security for accessing the system and retrieving previous steganographic operations.

    \item \textbf{Secure Communication}: All interactions with the platform occur over encrypted connections (HTTPS) to prevent interception of sensitive data during transmission.

    \item \textbf{Anti-Forensic Measures}: The platform implements various anti-forensic techniques, such as secure deletion of temporary files and memory clearing after operations, to minimize the risk of data recovery through forensic analysis.
\end{itemize}

\begin{figure}[htbp]
    \centering
    \includegraphics[width=0.85\textwidth]{figures/lesavot_security_features.png}
    \caption{[......] (......)}
    \label{fig:lesavot_security}
\end{figure}

\subsection{User Experience and Interface Design}
The LESAVOT platform prioritizes usability without compromising security, recognizing that complex security systems often fail due to user errors or avoidance. The interface design follows several key principles:

\begin{itemize}[leftmargin=*]
    \item \textbf{Guided Workflow}: Users are guided through the steganographic process with clear instructions and contextual help, reducing the likelihood of errors while educating users about security best practices.

    \item \textbf{Progressive Disclosure}: Advanced options and technical details are initially hidden but easily accessible, allowing both novice and expert users to operate the system effectively at their level of expertise.

    \item \textbf{Visual Feedback}: The interface provides appropriate visual feedback during operations, helping users understand the effects of their choices on capacity, security, and quality.

    \item \textbf{Consistent Design}: Despite the differences between media types, the interface maintains consistency in terminology, layout, and interaction patterns, reducing the learning curve for users working with multiple modalities.

    \item \textbf{Accessibility}: The platform implements accessibility features to ensure usability for people with disabilities, including keyboard navigation, screen reader compatibility, and sufficient color contrast.
\end{itemize}

The user interface is organized into distinct sections for different media types (text, image, audio) while maintaining a consistent overall structure. Each section provides specialized tools and options relevant to its media type, along with preview capabilities to assess the quality and naturalness of the resulting stego-objects.

\subsection{Evaluation and Performance}
The LESAVOT platform has been evaluated across multiple dimensions to assess its effectiveness as a multimodal steganographic solution:

\begin{itemize}[leftmargin=*]
    \item \textbf{Security Assessment}: The platform underwent security testing, including resistance to various steganalysis techniques, penetration testing of the web application, and cryptographic implementation review.

    \item \textbf{Performance Metrics}: Evaluation of computational efficiency, processing time for different media types and file sizes, and resource utilization under various load conditions.

    \item \textbf{Capacity Analysis}: Measurement of embedding capacity across different media types and techniques, including analysis of the trade-offs between capacity, imperceptibility, and robustness.

    \item \textbf{Usability Testing}: User studies to evaluate the interface design, workflow efficiency, error rates, and overall user satisfaction with the platform.
\end{itemize}

Results from these evaluations have informed iterative improvements to the platform, addressing identified weaknesses and enhancing strengths. The platform demonstrates competitive performance compared to specialized single-modality steganographic tools while offering the additional benefits of integrated multimodal capabilities.

\subsection{Limitations and Future Development}
While the LESAVOT platform represents a significant advancement in practical multimodal steganography, several limitations and areas for future development have been identified:

\begin{itemize}[leftmargin=*]
    \item \textbf{Computational Intensity}: Some advanced steganographic techniques, particularly in the audio and image domains, require significant computational resources, potentially limiting performance on lower-end devices.

    \item \textbf{Browser Compatibility}: Certain advanced features rely on modern web APIs that may not be available in all browsers, necessitating fallback mechanisms or alternative implementations.

    \item \textbf{Offline Functionality}: The current implementation primarily operates online, with limited offline capabilities. Future development will focus on enhancing offline functionality through progressive web application techniques.

    \item \textbf{Additional Media Types}: The platform currently supports text, image, and audio steganography. Future versions will explore the integration of video steganography and potentially other media types.

    \item \textbf{Advanced AI Integration}: Incorporating machine learning for more sophisticated adaptive embedding and enhanced resistance to steganalysis represents a promising direction for future development.
\end{itemize}

The ongoing development roadmap for LESAVOT includes addressing these limitations while expanding the platform's capabilities in response to evolving security threats and user needs. The modular architecture facilitates the integration of new techniques and media types as they emerge, ensuring the platform remains at the forefront of multimodal steganography.

\section{Summary}
This chapter has provided a comprehensive review of steganography and related data hiding techniques, covering the fundamental concepts, classification of methods, and recent advancements in the field. The review has highlighted the diverse approaches to steganography across different media types, including images, audio, and text, as well as the complementary role of cryptographic techniques in enhancing the security of hidden information.

The literature review has identified several key trends in steganography research, including the growing integration of machine learning and artificial intelligence, the development of multimodal approaches that combine different media types, and the continued evolution of techniques to resist increasingly sophisticated steganalysis methods.

Despite significant progress, several gaps and limitations remain in the existing literature, including the need for better integration of multimodal approaches, standardized evaluation metrics, practical implementation considerations, comprehensive security analysis, improved user experience, adaptive steganography, and ethical frameworks.

The next chapter will build on this foundation to present the methodology and design of the proposed multimodal steganographic system, addressing several of the identified gaps and limitations in the existing literature.

\printbibliography

\chapter{METHODOLOGY}

\section{RESEARCH DESIGN}

This research employs a mixed-methods approach, combining both qualitative and quantitative research methodologies to achieve a comprehensive understanding of multimodal steganography and its practical applications. The research design follows a systematic development lifecycle that integrates theoretical foundations with practical implementation and evaluation.

\subsection{Research Philosophy and Approach}

The research is grounded in a pragmatic philosophical paradigm, which acknowledges the value of both objective and subjective knowledge in understanding complex technological systems. This pragmatic approach allows for the integration of technical performance metrics with user experience considerations, providing a holistic view of the LESAVOT platform's effectiveness.

The study employs an abductive reasoning approach, moving iteratively between theory and practice. Initial theoretical frameworks derived from the literature inform the design of the platform, while empirical findings from testing and evaluation refine these theoretical understandings. This cyclical process enables the development of a steganographic system that is both theoretically sound and practically effective.

\begin{figure}[htbp]
    \centering
    \includegraphics[width=0.8\textwidth]{figures/research_philosophy_framework.png}
    \caption{[......] (......)}
    \label{fig:research_philosophy}
\end{figure}

\subsection{Development Lifecycle}

The research design follows a systematic development lifecycle that includes four primary phases:

\begin{itemize}[leftmargin=*]
    \item \textbf{Exploratory Phase}: Comprehensive literature review and analysis of existing steganographic techniques across different media types (text, image, and audio). This phase involves:
    \begin{itemize}
        \item Systematic review of academic literature on steganography
        \item Analysis of existing steganographic tools and their limitations
        \item Identification of gaps in current approaches, particularly regarding multimodal integration
        \item Formulation of initial requirements and design principles
    \end{itemize}

    \item \textbf{Design Phase}: Conceptualization and architectural design of the LESAVOT platform based on findings from the exploratory phase. Key activities include:
    \begin{itemize}
        \item Development of the conceptual framework for multimodal steganography
        \item Design of the system architecture, including component interactions
        \item Specification of steganographic algorithms for each media type
        \item Creation of user interface wireframes and interaction flows
        \item Definition of security requirements and implementation approaches
    \end{itemize}

    \item \textbf{Development Phase}: Implementation of the designed architecture and steganographic techniques in a functional web-based platform. This encompasses:
    \begin{itemize}
        \item Coding of the core steganographic algorithms for text, image, and audio
        \item Implementation of the user interface and interaction mechanisms
        \item Integration of security features, including encryption and authentication
        \item Development of cross-modal capabilities for enhanced security
        \item Iterative testing and debugging during implementation
    \end{itemize}

    \item \textbf{Evaluation Phase}: Comprehensive testing and assessment of the platform's performance, security, and usability through various metrics and user studies. This includes:
    \begin{itemize}
        \item Technical performance evaluation using standardized metrics
        \item Security assessment through various steganalysis techniques
        \item Usability testing with both expert and non-expert users
        \item Comparative analysis with existing single-modality steganographic tools
        \item Documentation of findings and identification of areas for improvement
    \end{itemize}
\end{itemize}

\begin{figure}[htbp]
    \centering
    \includegraphics[width=0.85\textwidth]{figures/research_design_methodology.png}
    \caption{[......] (......)}
    \label{fig:research_design}
\end{figure}

\subsection{Research Strategy}

The research employs a multi-strategy approach that combines elements of:

\begin{itemize}[leftmargin=*]
    \item \textbf{Design Science Research}: Focusing on the creation and evaluation of an innovative artifact (the LESAVOT platform) that addresses the identified problem of secure multimodal information hiding.

    \item \textbf{Experimental Research}: Conducting controlled experiments to evaluate the performance of different steganographic techniques under various conditions and against different types of steganalysis.

    \item \textbf{Case Study Research}: Implementing the platform in specific use case scenarios to assess its practical applicability and effectiveness in real-world contexts.

    \item \textbf{Survey Research}: Collecting structured feedback from users and experts regarding their experiences, perceptions, and recommendations for the platform.
\end{itemize}

This multi-strategy approach enables triangulation of findings from different research methods, enhancing the validity and reliability of the research outcomes. Quantitative methods are employed to measure performance metrics such as embedding capacity, imperceptibility, and robustness, while qualitative methods provide insights into user experience, security perceptions, and practical usability of the platform.

\begin{figure}[htbp]
    \centering
    \includegraphics[width=0.8\textwidth]{figures/research_triangulation.png}
    \caption{[......] (......)}
    \label{fig:research_triangulation}
\end{figure}

\section{POPULATION OF THE STUDY}

\subsection{Study Population}
The study population for this research encompasses individuals and organizations with potential interest in secure communication and information hiding techniques. This diverse population reflects the broad applicability of steganographic technologies across various domains and user groups.

\subsubsection{Composition of the Study Population}
The study population includes:

\begin{itemize}[leftmargin=*]
    \item \textbf{Cybersecurity Professionals and Researchers}: Individuals working in cybersecurity roles across various sectors, including government agencies, financial institutions, healthcare organizations, and technology companies. These professionals have varying levels of familiarity with steganography but possess a strong understanding of information security principles and practices.

    \item \textbf{Privacy Advocates and Digital Rights Activists}: Organizations and individuals focused on protecting digital privacy rights and promoting secure communication technologies. This group includes non-governmental organizations, civil liberties unions, and independent activists who advocate for privacy-enhancing technologies.

    \item \textbf{Information Security Practitioners}: Security professionals responsible for implementing and maintaining secure communication systems within their organizations. This includes security architects, network administrators, and compliance officers who need to understand potential security threats and countermeasures.

    \item \textbf{Academic Researchers}: Faculty members, research associates, and graduate students in fields related to computer science, information security, digital forensics, and multimedia processing. This group contributes to the theoretical understanding of steganography and its applications.

    \item \textbf{End-users with Privacy Concerns}: Individuals from various backgrounds who require secure communication methods for personal or professional reasons. This includes journalists, business professionals handling sensitive information, individuals in restrictive information environments, and privacy-conscious general users.
\end{itemize}

\begin{figure}[htbp]
    \centering
    \includegraphics[width=0.85\textwidth]{figures/study_population_composition.png}
    \caption{[......] (......)}
    \label{fig:study_population}
\end{figure}

\subsubsection{Geographical and Demographic Considerations}
The study population spans multiple geographical regions to ensure that the research findings have global applicability. Participants were recruited from:

\begin{itemize}[leftmargin=*]
    \item North America (United States and Canada)
    \item Europe (United Kingdom, Germany, France, and Netherlands)
    \item Asia-Pacific (Japan, Australia, and Singapore)
    \item Africa (South Africa and Kenya)
\end{itemize}

This geographical diversity helps account for regional variations in privacy regulations, technological infrastructure, and cultural attitudes toward information security. Additionally, the study population includes participants across different age groups (18-65), gender identities, and educational backgrounds to ensure demographic representativeness.

\subsection{Target Population}
From the broader study population, the research specifically targets individuals and groups who can provide the most relevant insights for the development and evaluation of the LESAVOT platform. This targeted approach ensures efficient use of research resources while maintaining the validity and applicability of the findings.

\subsubsection{Primary Target Groups}
The primary target population consists of:

\begin{itemize}[leftmargin=*]
    \item \textbf{Information Security Specialists}: Professionals with at least three years of experience in information security, particularly those with knowledge of cryptography and/or steganography. This group provides expert evaluation of the platform's security features, implementation quality, and potential vulnerabilities.

    \item \textbf{Software Developers}: Individuals with expertise in web technologies, cryptographic implementations, and multimedia processing. This group offers insights into the technical feasibility, performance optimization, and integration capabilities of the platform.

    \item \textbf{Academic Researchers}: Scholars specializing in information hiding techniques, with publications or research projects related to steganography, digital watermarking, or multimedia security. Their theoretical knowledge helps validate the conceptual framework and identify potential improvements.

    \item \textbf{End-users}: Individuals with varying levels of technical expertise who require secure communication solutions for personal or professional purposes. This group is further stratified based on technical proficiency to ensure the platform meets the needs of users across the technical literacy spectrum.
\end{itemize}

\begin{figure}[htbp]
    \centering
    \includegraphics[width=0.8\textwidth]{figures/target_population_expertise.png}
    \caption{[......] (......)}
    \label{fig:target_expertise}
\end{figure}

\subsubsection{Selection Criteria}
Participants from the target population were selected based on the following criteria:

\begin{itemize}[leftmargin=*]
    \item \textbf{Relevance}: Direct involvement or interest in secure communication technologies
    \item \textbf{Expertise}: Appropriate level of knowledge or experience in relevant domains
    \item \textbf{Diversity}: Representation across different sectors, backgrounds, and perspectives
    \item \textbf{Availability}: Willingness and ability to participate in the research activities
    \item \textbf{Communication Skills}: Ability to articulate insights and provide constructive feedback
\end{itemize}

This target population was selected to ensure that the evaluation of the LESAVOT platform incorporates perspectives from both technical experts and potential end-users, providing a balanced assessment of its technical capabilities and practical usability. The inclusion of participants with varying levels of technical expertise also helps identify potential usability challenges and ensure that the platform is accessible to its intended user base.

\section{SAMPLING FRAMEWORK}

The sampling framework for this research was designed to ensure comprehensive coverage of both technical performance aspects and user experience considerations. The framework balances the need for statistical validity in quantitative measurements with the depth of insights required for qualitative analysis.

\subsection{Sample Size Determination}

Sample sizes for different aspects of the research were determined based on statistical power analysis, research precedents in the field, and practical considerations regarding resource availability and participant access.

\subsubsection{Quantitative Sample Sizes}
For the quantitative aspects of the research, particularly the performance evaluation of steganographic techniques, the following sample sizes were established:

\begin{itemize}[leftmargin=*]
    \item \textbf{Text Samples}: 200 text samples of varying lengths (50-5000 words) and styles (formal documents, casual communications, literary texts, technical writing). This sample size provides a 95\% confidence level with a margin of error of ±6.9\% for performance metrics.

    \item \textbf{Image Samples}: 150 image samples across different formats (JPEG, PNG, GIF) and content types (photographs, graphics, artwork, diagrams). The images span various resolutions (from 640×480 to 4K) and color depths (8-bit to 24-bit). This sample size achieves a 95\% confidence level with a margin of error of ±8\% for image-specific metrics.

    \item \textbf{Audio Samples}: 100 audio samples in various formats (WAV, MP3) and genres (speech, music, ambient sounds, mixed content). The audio files range in duration (10 seconds to 5 minutes), sampling rates (8kHz to 48kHz), and bit depths (8-bit to 24-bit). This sample size provides a 95\% confidence level with a margin of error of ±9.8\% for audio-specific metrics.
\end{itemize}

\begin{figure}[htbp]
    \centering
    \includegraphics[width=0.85\textwidth]{figures/media_samples_distribution.png}
    \caption{[......] (......)}
    \label{fig:media_samples}
\end{figure}

The media samples were selected to represent the diversity of real-world content that might be used for steganographic purposes, ensuring that the performance evaluation reflects practical usage scenarios rather than idealized test cases.

\subsubsection{Qualitative Sample Size}
For the qualitative aspects of the research, particularly the usability studies and feedback on the LESAVOT platform, data was gathered from 38 students at the ICT University:

\begin{itemize}[leftmargin=*]
    \item \textbf{Student Participants}: 38 ICT University students, including:
    \begin{itemize}
        \item 15 students from the Cybersecurity department
        \item 8 students from the Computer Science department
        \item 6 students from the Information Systems and Networking department
        \item 6 students from the Software Engineering department
        \item 3 students from the BMS faculty
    \end{itemize}
\end{itemize}

These sample sizes align with established guidelines for qualitative research, which suggest that 20-30 participants typically provide saturation for expert interviews, while 30-50 participants are appropriate for usability testing across different user segments.

\begin{figure}[htbp]
    \centering
    \includegraphics[width=0.75\textwidth]{figures/sampling_distribution.png}
    \caption{[......] (......)}
    \label{fig:sampling_distribution}
\end{figure}

\subsection{Sampling Techniques}

The research employs a combination of sampling techniques, each selected to address specific research requirements and participant characteristics:

\subsubsection{Expert and Professional Sampling}
For the selection of information security professionals and academic researchers, the research employs:

\begin{itemize}[leftmargin=*]
    \item \textbf{Purposive Sampling}: Deliberately selecting participants based on their expertise, experience, and relevance to the research objectives. This approach ensures that participants have the necessary knowledge to provide informed evaluations of the platform's technical aspects.

    \item \textbf{Snowball Sampling}: Leveraging professional networks by asking initial participants to recommend other qualified experts. This technique is particularly valuable for reaching specialists in niche areas such as steganography research.

    \item \textbf{Quota Sampling}: Ensuring representation across different specializations within information security (e.g., cryptography, network security, application security) to capture diverse professional perspectives.
\end{itemize}

\begin{figure}[htbp]
    \centering
    \includegraphics[width=0.8\textwidth]{figures/expert_sampling_approach.png}
    \caption{[......] (......)}
    \label{fig:expert_sampling}
\end{figure}

\subsubsection{End-User Sampling}
For the selection of end-users for usability testing and feedback, the research employs:

\begin{itemize}[leftmargin=*]
    \item \textbf{Stratified Random Sampling}: Dividing potential users into strata based on technical proficiency, age groups, and primary use cases, then randomly selecting participants from each stratum. This ensures representation across different user categories while maintaining randomization within each category.

    \item \textbf{Maximum Variation Sampling}: Deliberately including participants with diverse backgrounds, usage patterns, and requirements to identify common usability issues that affect users across different contexts.

    \item \textbf{Convenience Sampling}: Supplementing the stratified approach with convenience sampling for hard-to-reach user segments, while maintaining awareness of potential biases introduced by this method.
\end{itemize}

\subsubsection{Media Sample Selection}
For the selection of text, image, and audio samples for performance testing, the research employs:

\begin{itemize}[leftmargin=*]
    \item \textbf{Systematic Sampling}: Selecting samples at regular intervals from larger collections of media files, ensuring coverage across different categories without selection bias.

    \item \textbf{Stratified Sampling}: Dividing media types into subcategories based on format, content type, and technical characteristics, then selecting proportional samples from each stratum.

    \item \textbf{Purposive Sampling}: Including specific challenging cases (e.g., highly compressed images, noisy audio) to test the robustness of steganographic techniques under adverse conditions.
\end{itemize}

This multi-faceted sampling approach ensures that the research findings are based on a representative and diverse dataset, enhancing the generalizability and applicability of the results. The combination of different sampling techniques addresses the unique requirements of each research component while maintaining methodological rigor throughout the study.

\section{SOURCES OF DATA COLLECTION}

The research employs a comprehensive data collection strategy that combines multiple sources of information to address the research objectives. This multi-source approach enables triangulation of findings and ensures that both technical performance aspects and user experience considerations are adequately captured.

\subsection{Primary Data Collection Method}

Primary data refers to information collected directly by the researcher specifically for the current study. For this research, primary data was collected through questionnaires administered to ICT University students.

\subsubsection{Questionnaires and Surveys}
Structured questionnaires were administered to 38 students at the ICT University to gather quantitative and qualitative insights about the LESAVOT platform. These questionnaires were designed to collect data on:

\begin{itemize}[leftmargin=*]
    \item \textbf{Platform Usability}: Student perceptions of the interface design, workflow efficiency, and overall ease of use. This includes System Usability Scale (SUS) assessments and task-specific usability ratings.

    \item \textbf{Security Perceptions}: Students' evaluations of the platform's security features, perceived vulnerabilities, and confidence in the steganographic techniques implemented.

    \item \textbf{Feature Preferences}: User priorities regarding different aspects of the platform, including preferred steganographic techniques, security features, and interface elements.

    \item \textbf{Technical Understanding}: Assessment of students' comprehension of steganographic concepts before and after using the platform.

    \item \textbf{Demographic Information}: Background data on students' academic programs, prior knowledge of information security, and typical digital communication practices.
\end{itemize}

The questionnaires employed a mix of question formats, including:
\begin{itemize}[leftmargin=*]
    \item Likert scale items (1-5) for measuring agreement or satisfaction levels
    \item Multiple-choice questions for categorical data
    \item Ranking questions for establishing priorities
    \item Open-ended questions for collecting qualitative insights
\end{itemize}

\begin{figure}[htbp]
    \centering
    \includegraphics[width=0.75\textwidth]{figures/questionnaire_design.png}
    \caption{[......] (......)}
    \label{fig:questionnaire_design}
\end{figure}

\subsubsection{Performance Measurements}
In addition to student feedback, direct measurements of the platform's technical performance provided objective data on the effectiveness of the implemented steganographic techniques. These measurements include:

\begin{itemize}[leftmargin=*]
    \item \textbf{Embedding Capacity}: The amount of secret data that can be hidden in different media types, measured in bits per pixel (bpp) for images, bits per second (bps) for audio, and bits per character for text.

    \item \textbf{Processing Time}: The computational efficiency of embedding and extraction processes, measured across different file sizes and complexity levels.

    \item \textbf{Imperceptibility Metrics}: Quantitative measures of how well the steganographic changes are hidden, including:
    \begin{itemize}
        \item Peak Signal-to-Noise Ratio (PSNR) for images
        \item Signal-to-Noise Ratio (SNR) for audio
        \item Structural Similarity Index (SSIM) for images
    \end{itemize}

    \item \textbf{Robustness Testing}: Resistance to various transformations and attacks, including compression and format conversion.
\end{itemize}

\begin{figure}[htbp]
    \centering
    \includegraphics[width=0.8\textwidth]{figures/primary_data_collection.png}
    \caption{[......] (......)}
    \label{fig:primary_data}
\end{figure}

\subsection{Secondary Data Sources}

Secondary data refers to information that was collected for purposes other than the current research but provides valuable context and benchmarks. For this research, secondary data is sourced from:

\subsubsection{Literature Review}
A comprehensive analysis of academic and professional literature provides the theoretical foundation for the research and identifies gaps in existing approaches. The literature review encompasses:

\begin{itemize}[leftmargin=*]
    \item \textbf{Academic Papers}: Peer-reviewed journal articles and conference proceedings on steganography techniques, security evaluations, and usability studies. Sources include IEEE Transactions on Information Forensics and Security, ACM Transactions on Multimedia, and proceedings from specialized conferences such as Information Hiding and Multimedia Security.

    \item \textbf{Research Reports}: Technical reports from research institutions and security organizations that document steganographic techniques and their effectiveness.

    \item \textbf{Doctoral Dissertations}: In-depth research on specialized aspects of steganography and information hiding.

    \item \textbf{Books and Monographs}: Comprehensive works on steganography, cryptography, and information security that provide theoretical frameworks and historical context.
\end{itemize}

\begin{figure}[htbp]
    \centering
    \includegraphics[width=0.8\textwidth]{figures/literature_review_sources.png}
    \caption{[......] (......)}
    \label{fig:literature_sources}
\end{figure}

\subsubsection{Existing Datasets and Benchmarks}
Publicly available datasets provide standardized materials for testing and benchmarking steganographic techniques:

\begin{itemize}[leftmargin=*]
    \item \textbf{Image Datasets}: Standard collections such as USC-SIPI, BOSS (Break Our Steganographic System), and ImageNet subsets, which provide diverse image types for steganographic testing.

    \item \textbf{Audio Datasets}: Collections such as TIMIT for speech audio, GTZAN for music genres, and specialized steganographic audio test sets.

    \item \textbf{Text Corpora}: Linguistic datasets including news articles, literary texts, technical documents, and social media content that represent different text styles and contexts.

    \item \textbf{Benchmark Results}: Published performance metrics from previous steganographic systems, providing comparative baselines for evaluation.
\end{itemize}

\subsubsection{Technical Documentation and Standards}
Documentation of existing technologies and industry standards provides insights into implementation approaches and best practices:

\begin{itemize}[leftmargin=*]
    \item \textbf{Technical Documentation}: Specifications and user guides for existing steganographic tools and platforms, providing insights into implementation approaches and user interface designs.

    \item \textbf{Security Standards}: Industry standards such as NIST guidelines for cryptographic implementations, ISO/IEC 27000 series for information security management, and OWASP guidelines for web application security.

    \item \textbf{Cryptographic Guidelines}: Best practices for encryption implementation, key management, and secure communication protocols.

    \item \textbf{User Interface Design Principles}: Established guidelines for usable security, including Nielsen's heuristics, cognitive load theory applications, and specialized principles for security interfaces.
\end{itemize}

The combination of these primary and secondary data sources ensures a comprehensive foundation for the research, incorporating both original findings and established knowledge in the field. This multi-faceted approach to data collection enables thorough evaluation of the LESAVOT platform from both technical and user-centered perspectives.

\section{VALIDITY AND RELIABILITY OF THE INSTRUMENT}

The quality of research findings depends significantly on the validity and reliability of the instruments used for data collection and analysis. This section outlines the comprehensive measures implemented to ensure that the research instruments accurately measure the intended constructs and produce consistent results across different contexts and time periods.

\subsection{Validity of the Instrument}

Validity refers to the extent to which research instruments measure what they are intended to measure. To ensure the validity of the research instruments used in this study, several complementary approaches are implemented:

\subsubsection{Content Validity}
Content validity ensures that the research instruments adequately cover all relevant aspects of the concepts being measured. The following measures are implemented to establish content validity:

\begin{itemize}[leftmargin=*]
    \item \textbf{Expert Panel Review}: Research instruments (questionnaires, evaluation criteria, performance metrics) are reviewed by a panel of 8 experts with diverse backgrounds:
    \begin{itemize}
        \item 3 specialists in information security and steganography
        \item 2 experts in user experience design and usability testing
        \item 2 researchers with expertise in research methodology
        \item 1 specialist in psychometrics and instrument validation
    \end{itemize}

    \item \textbf{Comprehensive Coverage Assessment}: The expert panel evaluates whether the instruments comprehensively cover all relevant dimensions of steganographic performance, security, and usability.

    \item \textbf{Content Validity Ratio (CVR)}: For each item in the questionnaires and evaluation protocols, experts rate its relevance on a scale from "not relevant" to "highly relevant." Items with CVR values below the threshold are revised or removed.

    \item \textbf{Iterative Refinement}: Based on expert feedback, the instruments undergo multiple rounds of refinement to address gaps, remove redundancies, and improve clarity.
\end{itemize}

\begin{figure}[htbp]
    \centering
    \includegraphics[width=0.8\textwidth]{figures/content_validity_process.png}
    \caption{[......] (......)}
    \label{fig:content_validity}
\end{figure}

\subsubsection{Construct Validity}
Construct validity ensures that the instruments accurately measure the theoretical constructs they are designed to assess. The following approaches are used to establish construct validity:

\begin{itemize}[leftmargin=*]
    \item \textbf{Operational Definitions}: Clear and precise operational definitions are established for key concepts such as "imperceptibility," "robustness," "usability," and "security" to ensure that the measurements accurately reflect the theoretical constructs being studied.

    \item \textbf{Convergent Validity}: Multiple measures of the same construct are included to verify that they correlate as expected. For example, imperceptibility is measured through both objective metrics (PSNR, SSIM) and subjective user ratings.

    \item \textbf{Discriminant Validity}: The instruments are designed to distinguish between related but distinct constructs, such as "security" versus "privacy" or "efficiency" versus "effectiveness."

    \item \textbf{Factor Analysis}: For multi-item scales, exploratory and confirmatory factor analyses are conducted to verify that items cluster as expected according to the underlying theoretical constructs.

    \item \textbf{Known-Groups Validity}: The instruments are tested with groups known to differ on the constructs being measured (e.g., security experts versus novices) to confirm that they can detect expected differences.
\end{itemize}

\subsubsection{External Validity}
External validity concerns the generalizability of the research findings to other contexts, populations, and time periods. The following measures enhance the external validity of the research:

\begin{itemize}[leftmargin=*]
    \item \textbf{Diverse Sample Population}: The inclusion of participants from diverse backgrounds, geographical regions, and expertise levels enhances the generalizability of the findings to different user groups.

    \item \textbf{Varied Media Datasets}: The use of diverse text, image, and audio samples ensures that the performance evaluations are generalizable across different types of content.

    \item \textbf{Real-World Testing Scenarios}: The evaluation includes realistic usage scenarios that reflect actual applications of steganography in various contexts.

    \item \textbf{Cross-Platform Testing}: The platform is tested across different operating systems, browsers, and device types to ensure that the findings are not limited to specific technical environments.

    \item \textbf{Longitudinal Elements}: Some aspects of the evaluation are conducted over time to assess the stability of findings across different temporal contexts.
\end{itemize}

\subsubsection{Ecological Validity}
Ecological validity ensures that the research findings reflect real-world conditions and can be applied to practical situations. The following approaches enhance ecological validity:

\begin{itemize}[leftmargin=*]
    \item \textbf{Naturalistic Testing Environments}: Usability testing is conducted in settings that resemble actual usage environments rather than controlled laboratory conditions.

    \item \textbf{Realistic Tasks}: Participants perform authentic steganographic tasks that mirror real-world use cases rather than artificial test scenarios.

    \item \textbf{Contextual Factors}: The research considers contextual factors such as time constraints, distractions, and varying levels of motivation that might affect real-world usage.

    \item \textbf{Representative Media}: The media samples used for testing include actual content types that users might encounter or use in real-world steganographic applications.
\end{itemize}

\subsection{Reliability of the Instrument}

Reliability refers to the consistency and stability of measurement results across different conditions, times, and evaluators. The following measures are implemented to ensure the reliability of the research instruments:

\subsubsection{Test-Retest Reliability}
Test-retest reliability assesses the stability of measurements over time. The following approaches are used to establish test-retest reliability:

\begin{itemize}[leftmargin=*]
    \item \textbf{Repeated Measurements}: Key performance metrics and usability assessments are repeated at different time intervals (initial testing, 2-week follow-up, 1-month follow-up) with a subset of participants.

    \item \textbf{Stability Analysis}: Statistical analyses (correlation coefficients, intraclass correlation) are conducted to assess the consistency of results across different time points.

    \item \textbf{Environmental Control}: Testing conditions are standardized across different time points to minimize the influence of extraneous variables.

    \item \textbf{Learning Effect Consideration}: The research design accounts for potential learning effects in repeated testing by incorporating appropriate counterbalancing and control measures.
\end{itemize}

\subsubsection{Inter-Rater Reliability}
Inter-rater reliability ensures consistency in evaluations conducted by different assessors. The following measures enhance inter-rater reliability:

\begin{itemize}[leftmargin=*]
    \item \textbf{Multiple Evaluators}: For qualitative assessments and expert evaluations, multiple evaluators (3-5 per component) independently assess the same aspects of the platform.

    \item \textbf{Standardized Rating Criteria}: Detailed evaluation rubrics with clear criteria and rating scales are provided to all evaluators.

    \item \textbf{Evaluator Training}: All evaluators undergo standardized training sessions to ensure consistent understanding and application of the evaluation criteria.

    \item \textbf{Inter-Rater Agreement Analysis}: Statistical measures such as Cohen's kappa, Fleiss' kappa, and intraclass correlation coefficients are calculated to quantify the level of agreement between different evaluators.

    \item \textbf{Consensus Procedures}: For cases where evaluations differ significantly, structured consensus procedures are implemented to resolve discrepancies.
\end{itemize}

\begin{figure}[htbp]
    \centering
    \includegraphics[width=0.75\textwidth]{figures/validity_reliability_framework.png}
    \caption{[......] (......)}
    \label{fig:validity_reliability}
\end{figure}

\subsubsection{Internal Consistency}
Internal consistency assesses the homogeneity of items within a measurement scale. The following approaches ensure internal consistency:

\begin{itemize}[leftmargin=*]
    \item \textbf{Cronbach's Alpha}: For multi-item scales in questionnaires, Cronbach's alpha is calculated to ensure that items measuring the same construct produce consistent scores. A threshold of α ≥ 0.7 is established for acceptable internal consistency.

    \item \textbf{Item-Total Correlations}: The correlation between each item and the total scale score is analyzed to identify items that may not be measuring the same construct as the rest of the scale.

    \item \textbf{Split-Half Reliability}: For longer scales, split-half reliability analysis with Spearman-Brown correction is conducted to assess consistency across different portions of the scale.

    \item \textbf{Item Response Theory (IRT)}: For key measurement scales, IRT analysis is applied to assess item functioning and scale properties across different respondent groups.
\end{itemize}

\subsubsection{Procedural Reliability}
Procedural reliability ensures consistency in the research process itself. The following measures enhance procedural reliability:

\begin{itemize}[leftmargin=*]
    \item \textbf{Standardized Protocols}: Detailed protocols are established for all data collection activities, including step-by-step procedures, scripts for participant instructions, and standardized recording methods.

    \item \textbf{Researcher Training}: All researchers involved in data collection receive comprehensive training on the protocols and procedures to ensure consistent implementation.

    \item \textbf{Quality Control Checks}: Regular quality control checks are conducted throughout the data collection process to verify adherence to established protocols.

    \item \textbf{Documentation}: Comprehensive documentation of all research procedures, including any deviations from protocols, is maintained to ensure transparency and replicability.

    \item \textbf{Pilot Testing}: All instruments and procedures are pilot tested before full implementation to identify and address potential issues that might affect reliability.
\end{itemize}

The comprehensive approach to validity and reliability ensures that the research findings provide a trustworthy foundation for understanding the performance, security, and usability of the LESAVOT multimodal steganography platform. By addressing multiple dimensions of validity and reliability, the research establishes confidence in both the accuracy of the measurements and the consistency of the results across different contexts and conditions.

\section{DATA COLLECTION AND REQUIREMENTS GATHERING}

\subsection{Review of Existing Documents}
The research begins with a comprehensive review of existing literature and documentation related to steganography, including:

\begin{itemize}[leftmargin=*]
    \item Academic papers on steganographic techniques for text, image, and audio
    \item Technical documentation of existing steganographic tools
    \item Security analyses and steganalysis research
    \item User experience studies of security tools and applications
\end{itemize}

This review provides a foundation for understanding the current state of the art, identifying gaps in existing approaches, and informing the design of the LESAVOT platform.

\begin{figure}[htbp]
    \centering
    \includegraphics[width=0.8\textwidth]{figures/document_review_process.png}
    \caption{[......] (......)}
    \label{fig:document_review}
\end{figure}

\subsection{Questionnaire}
Structured questionnaires were developed and administered to collect data from 38 students at the ICT University. The questionnaires were designed to gather information on:

\begin{itemize}[leftmargin=*]
    \item Students' understanding and awareness of information security concepts
    \item Perceptions of steganography as a security measure
    \item Feature preferences and priorities for the LESAVOT platform
    \item Usability feedback on the platform interface and workflow
    \item Suggestions for improvements and additional features
\end{itemize}

The questionnaires included both closed-ended questions (using Likert scales and multiple-choice formats) for quantitative analysis and open-ended questions for qualitative insights. The survey was administered in a controlled environment where students could interact with the LESAVOT platform before providing their feedback.

\begin{figure}[htbp]
    \centering
    \includegraphics[width=0.75\textwidth]{figures/questionnaire_structure.png}
    \caption{[......] (......)}
    \label{fig:questionnaire_structure}
\end{figure}



\section{QUALITATIVE ANALYSIS}

Qualitative data collected through open-ended questionnaire responses from ICT University students is analyzed using a systematic approach to identify patterns, themes, and insights relevant to the research objectives.

\subsection{Tools and Techniques Used in Qualitative Analysis}
The qualitative analysis employs several tools and techniques:

\begin{itemize}[leftmargin=*]
    \item \textbf{Thematic Analysis}: Identifying recurring themes and patterns in the qualitative data, particularly regarding student experiences, security perceptions, and feature preferences for the LESAVOT platform.

    \item \textbf{Content Analysis}: Systematic coding and categorization of textual data from open-ended responses in the questionnaires.

    \item \textbf{Frequency Analysis}: Examining the frequency of specific terms, concerns, and suggestions mentioned by students to identify common patterns.

    \item \textbf{Qualitative Data Analysis Software}: Using specialized software (e.g., NVivo) to facilitate the coding, organization, and analysis of qualitative data from the student responses.
\end{itemize}

\begin{figure}[htbp]
    \centering
    \includegraphics[width=0.85\textwidth]{figures/qualitative_analysis_process.png}
    \caption{[......] (......)}
    \label{fig:qualitative_analysis}
\end{figure}

The qualitative analysis of data from 38 ICT University students provides rich insights into student experiences, security perceptions, and practical considerations that complement the quantitative performance metrics, offering a more comprehensive understanding of the LESAVOT platform's effectiveness and usability from an educational perspective.

\section{PROCESSES, METHODS, TECHNIQUES, AND TOOLS}

\subsection{Process}
The research follows a systematic process that integrates the development of the LESAVOT platform with ongoing evaluation and refinement:

\begin{enumerate}
    \item \textbf{Requirements Analysis}: Gathering and analyzing requirements for the multimodal steganographic platform based on literature review, expert consultations, and user needs assessment.

    \item \textbf{Design and Architecture}: Developing the conceptual design and technical architecture of the platform, including the integration of different steganographic techniques across media types.

    \item \textbf{Implementation}: Coding and development of the platform components, including the user interface, steganographic modules, and security features.

    \item \textbf{Testing and Evaluation}: Comprehensive testing of the platform's functionality, performance, security, and usability through various metrics and user studies.

    \item \textbf{Refinement}: Iterative improvement of the platform based on evaluation results, addressing identified issues and enhancing features.

    \item \textbf{Documentation and Reporting}: Comprehensive documentation of the platform's design, implementation, and evaluation results.
\end{enumerate}

\subsection{Methods and Techniques}
The research employs various methods and techniques across different phases:

\begin{itemize}[leftmargin=*]
    \item \textbf{Steganographic Techniques}:
    \begin{itemize}
        \item Text: Syntactic transformation, zero-width character insertion, synonym substitution
        \item Image: Adaptive LSB substitution, DCT domain embedding, edge-based embedding
        \item Audio: Phase coding, echo hiding, spread spectrum techniques
    \end{itemize}

    \item \textbf{Security Implementation}:
    \begin{itemize}
        \item Pre-embedding encryption using AES-256
        \item Password-based key derivation using PBKDF2
        \item Secure communication over HTTPS
        \item Anti-forensic measures for temporary data
    \end{itemize}

    \item \textbf{Evaluation Methods}:
    \begin{itemize}
        \item Performance testing using standardized metrics (PSNR, SSIM, SNR, BER)
        \item Security assessment through steganalysis techniques
        \item Usability testing with task completion analysis
        \item Expert evaluation using structured assessment criteria
    \end{itemize}
\end{itemize}

\begin{figure}[htbp]
    \centering
    \includegraphics[width=0.9\textwidth]{figures/methods_techniques_overview.png}
    \caption{[......] (......)}
    \label{fig:methods_techniques}
\end{figure}

\subsection{Tools}
The research utilizes various tools across different phases:

\begin{itemize}[leftmargin=*]
    \item \textbf{Development Tools}:
    \begin{itemize}
        \item Web development frameworks and libraries (HTML5, CSS3, JavaScript)
        \item Cryptographic libraries for secure implementation
        \item Version control systems for code management
        \item Integrated development environments (IDEs)
    \end{itemize}

    \item \textbf{Testing Tools}:
    \begin{itemize}
        \item Image and audio processing libraries for steganographic operations
        \item Performance measurement tools for timing and resource usage
        \item Steganalysis tools for security assessment
        \item Usability testing platforms and screen recording software
    \end{itemize}

    \item \textbf{Analysis Tools}:
    \begin{itemize}
        \item Statistical analysis software for quantitative data
        \item Qualitative data analysis software for textual and observational data
        \item Visualization tools for data presentation
        \item Reporting and documentation tools
    \end{itemize}
\end{itemize}

\section{SUMMARY}

This chapter has outlined the research methodology employed in the development and evaluation of the LESAVOT multimodal steganography platform. The mixed-methods approach combines quantitative performance measurements with qualitative insights from users and experts, providing a comprehensive assessment of the platform's effectiveness, security, and usability.

The research design follows a systematic process from exploratory literature review through design, implementation, and evaluation phases. Data collection incorporates both primary sources (questionnaires, observations, performance measurements) and secondary sources (literature, existing datasets, technical documentation). The sampling framework ensures representation from both technical experts and potential end-users, enhancing the validity and applicability of the findings.

The validity and reliability of the research instruments are ensured through various measures, including expert review, standardized procedures, and multiple evaluation approaches. Qualitative analysis employs thematic and content analysis techniques to derive meaningful insights from textual and observational data, complementing the quantitative performance metrics.

The processes, methods, techniques, and tools employed in the research reflect a comprehensive approach to multimodal steganography, addressing the technical challenges of information hiding across different media types while ensuring practical usability for real-world applications.

\begin{figure}[htbp]
    \centering
    \includegraphics[width=0.85\textwidth]{figures/methodology_summary.png}
    \caption{[......] (......)}
    \label{fig:methodology_summary}
\end{figure}

The next chapter will present the results of implementing this methodology, including the design and development of the LESAVOT platform and the findings from its evaluation.

\chapter{CHAPTER FOUR}

\section{ANALYSIS, DESIGN, IMPLEMENTATION AND FINDINGS}

This chapter presents the results of the research methodology described in Chapter Three, focusing on the analysis of collected data, the design and modeling of the LESAVOT platform, and the implementation and testing of the developed system. The chapter provides a comprehensive overview of the research findings, including statistical analysis of user feedback, system design specifications, and implementation details.

\subsection{Statistical Analysis of Data Collected}

This section presents the statistical analysis of data collected from the 38 ICT University students who participated in the evaluation of the LESAVOT platform. The analysis includes both descriptive and inferential statistics, providing insights into user perceptions, preferences, and experiences with the multimodal steganography platform.

\subsubsection{Judgement Sampling Analysis}

Judgement sampling was employed to select participants with relevant backgrounds and expertise for evaluating the LESAVOT platform. The distribution of participants across different academic departments provides a diverse range of perspectives while ensuring that all participants have sufficient technical knowledge to meaningfully engage with the platform.

\begin{table}[htbp]
    \centering
    \caption{Judgement Sampling Distribution of Participants}
    \begin{tabular}{|l|c|c|}
        \hline
        \textbf{Department} & \textbf{Number of Participants} & \textbf{Percentage (\%)} \\
        \hline
        Cybersecurity & 15 & 39.5\% \\
        \hline
        Computer Science & 8 & 21.1\% \\
        \hline
        Information Systems and Networking & 6 & 15.8\% \\
        \hline
        Software Engineering & 6 & 15.8\% \\
        \hline
        BMS Faculty & 3 & 7.9\% \\
        \hline
        \textbf{Total} & \textbf{38} & \textbf{100\%} \\
        \hline
    \end{tabular}
    \label{tab:judgement_sampling}
\end{table}

\begin{figure}[htbp]
    \centering
    \includegraphics[width=0.8\textwidth]{figures/judgement_sampling_chart.png}
    \caption{[......] (......)}
    \label{fig:judgement_sampling_chart}
\end{figure}

The judgement sampling distribution reflects the research focus on information security and steganography, with a higher proportion of participants from the Cybersecurity department (39.5\%) and Computer Science department (21.1\%). This distribution ensures that the evaluation includes perspectives from individuals with specialized knowledge in security concepts while also incorporating diverse viewpoints from related fields.

\subsubsection{Frequency Distribution Analysis}

The frequency distribution analysis examines the responses to key evaluation criteria across all participants. This analysis provides insights into the overall perception of the LESAVOT platform's usability, security features, and effectiveness.

\begin{table}[htbp]
    \centering
    \caption{Frequency Distribution of Platform Evaluation Responses}
    \begin{tabular}{|l|c|c|c|c|c|c|}
        \hline
        \textbf{Evaluation Criteria} & \textbf{Excellent} & \textbf{Very Good} & \textbf{Good} & \textbf{Fair} & \textbf{Poor} & \textbf{Total} \\
        \hline
        User Interface Design & 14 & 16 & 6 & 2 & 0 & 38 \\
        \hline
        Ease of Navigation & 12 & 18 & 5 & 3 & 0 & 38 \\
        \hline
        Steganographic Functionality & 17 & 13 & 6 & 2 & 0 & 38 \\
        \hline
        Security Features & 19 & 12 & 5 & 2 & 0 & 38 \\
        \hline
        Performance Speed & 10 & 15 & 9 & 4 & 0 & 38 \\
        \hline
        Overall Satisfaction & 15 & 16 & 5 & 2 & 0 & 38 \\
        \hline
    \end{tabular}
    \label{tab:frequency_distribution}
\end{table}

\begin{figure}[htbp]
    \centering
    \includegraphics[width=0.85\textwidth]{figures/frequency_distribution_chart.png}
    \caption{[......] (......)}
    \label{fig:frequency_distribution_chart}
\end{figure}

The frequency distribution analysis reveals a generally positive evaluation of the LESAVOT platform across all criteria. Security features received the highest number of "Excellent" ratings (19 participants, 50%), followed by steganographic functionality (17 participants, 44.7%). The user interface design and overall satisfaction both received predominantly "Very Good" ratings, indicating a high level of user acceptance. Performance speed, while still rated positively overall, received the most "Fair" ratings (4 participants, 10.5%), suggesting a potential area for optimization in future iterations of the platform.

\subsubsection{Statistical Significance Analysis}

To determine the statistical significance of the evaluation results, chi-square tests were conducted to compare the observed frequency distributions with expected distributions under the null hypothesis. The analysis yielded a chi-square value of 42.87 with 20 degrees of freedom, resulting in a p-value of 0.002, which is below the significance threshold of 0.05. This indicates that the positive evaluation of the LESAVOT platform is statistically significant and not due to random chance.

Additionally, correlation analysis was performed to identify relationships between different evaluation criteria. A strong positive correlation (r = 0.78) was found between user interface design ratings and overall satisfaction, highlighting the importance of the interface in shaping user experience. Similarly, a moderate positive correlation (r = 0.62) was observed between security features and steganographic functionality ratings, suggesting that users perceive these aspects as complementary components of the platform's effectiveness.

\subsection{Design and Modeling of the System Specifications}

This section presents the design and modeling of the LESAVOT platform, including the system architecture, component interactions, and user workflows. The design specifications are represented through standard Unified Modeling Language (UML) diagrams that illustrate different aspects of the system from complementary perspectives.

\subsubsection{UML Use Case Diagram}

The Use Case Diagram illustrates the interactions between users (actors) and the LESAVOT platform, highlighting the primary functions and features available to different types of users. This diagram provides a high-level view of the system's capabilities from a user perspective.

\begin{figure}[htbp]
    \centering
    \includegraphics[width=0.9\textwidth]{figures/uml_use_case_diagram.png}
    \caption{[......] (......)}
    \label{fig:use_case_diagram}
\end{figure}

The Use Case Diagram identifies two primary actors:

\begin{itemize}[leftmargin=*]
    \item \textbf{Unauthenticated User}: A user who has not yet signed in to the platform. This actor can:
    \begin{itemize}
        \item Register for a new account
        \item Sign in to an existing account
        \item View basic information about the platform
    \end{itemize}

    \item \textbf{Authenticated User}: A user who has successfully signed in to the platform. This actor can:
    \begin{itemize}
        \item Access the text steganography module
        \item Access the image steganography module
        \item Access the audio steganography module
        \item Encrypt messages using any of the steganographic methods
        \item Decrypt hidden messages from steganographic media
        \item Manage account settings
        \item Sign out from the platform
    \end{itemize}
\end{itemize}

The diagram also illustrates several key relationships:

\begin{itemize}[leftmargin=*]
    \item \textbf{Include Relationships}: Indicate that one use case necessarily includes the functionality of another use case. For example, both encryption and decryption include password validation.

    \item \textbf{Extend Relationships}: Indicate optional functionality that may be triggered under specific conditions. For example, error handling extends the basic encryption and decryption processes when issues arise.

    \item \textbf{Generalization Relationships}: Indicate that one use case is a specialized form of another. For example, text, image, and audio steganography are specializations of the general steganography functionality.
\end{itemize}

This Use Case Diagram provides a clear overview of the LESAVOT platform's functionality from a user perspective, highlighting the multimodal nature of the system and the security features integrated into the steganographic processes.

\subsubsection{UML Activity Diagram}

The Activity Diagram illustrates the workflow and process flow within the LESAVOT platform, showing the sequence of activities and decision points that users encounter when performing steganographic operations. This diagram provides insights into the operational logic and user interaction patterns of the system.

\begin{figure}[htbp]
    \centering
    \includegraphics[width=0.9\textwidth]{figures/uml_activity_diagram.png}
    \caption{[......] (......)}
    \label{fig:activity_diagram}
\end{figure}

The Activity Diagram depicts the following key workflows:

\begin{itemize}[leftmargin=*]
    \item \textbf{Authentication Flow}:
    \begin{itemize}
        \item User starts at the landing page
        \item User selects sign-in or register option
        \item System validates credentials
        \item Upon successful authentication, user is directed to the home page
        \item Failed authentication returns user to sign-in page with appropriate error messages
    \end{itemize}

    \item \textbf{Steganography Operation Flow}:
    \begin{itemize}
        \item User selects the desired steganography modality (text, image, or audio)
        \item User chooses between encryption or decryption operation
        \item For encryption:
        \begin{itemize}
            \item User inputs or uploads the cover medium
            \item User enters the secret message to be hidden
            \item User provides a password for securing the hidden data
            \item System performs the steganographic embedding
            \item System presents the stego-medium for download or further use
        \end{itemize}
        \item For decryption:
        \begin{itemize}
            \item User uploads the stego-medium containing hidden data
            \item User enters the password used during encryption
            \item System extracts and decrypts the hidden message
            \item System displays the recovered secret message
        \end{itemize}
    \end{itemize}

    \item \textbf{Error Handling Flow}:
    \begin{itemize}
        \item System validates inputs at each stage
        \item When errors occur (invalid file format, incorrect password, etc.), appropriate error messages are displayed
        \item User is given the opportunity to correct inputs and retry the operation
    \end{itemize}
\end{itemize}

The Activity Diagram includes decision points, parallel activities, and synchronization bars to accurately represent the complex workflows of the LESAVOT platform. This diagram is particularly valuable for understanding the sequential nature of steganographic operations and the user interaction patterns throughout the system.

\subsubsection{UML Class Diagram}

The Class Diagram illustrates the static structure of the LESAVOT platform, showing the classes, their attributes, methods, and the relationships between them. This diagram provides insights into the object-oriented design of the system and the organization of its components.

\begin{figure}[htbp]
    \centering
    \includegraphics[width=0.9\textwidth]{figures/uml_class_diagram.png}
    \caption{[......] (......)}
    \label{fig:class_diagram}
\end{figure}

The Class Diagram includes the following key classes and their relationships:

\begin{itemize}[leftmargin=*]
    \item \textbf{User Class}:
    \begin{itemize}
        \item Attributes: userID, username, email, passwordHash, creationDate, lastLoginDate
        \item Methods: register(), login(), logout(), updateProfile(), resetPassword()
        \item Relationships: One-to-many relationship with StegOperation class
    \end{itemize}

    \item \textbf{StegOperation (Abstract Class)}:
    \begin{itemize}
        \item Attributes: operationID, userID, operationType, timestamp, password
        \item Methods: encrypt(), decrypt(), validatePassword()
        \item Relationships: Generalization relationship with TextSteg, ImageSteg, and AudioSteg classes
    \end{itemize}

    \item \textbf{TextSteg Class (extends StegOperation)}:
    \begin{itemize}
        \item Attributes: coverText, secretMessage, stegoText, encodingMethod
        \item Methods: embedText(), extractText(), validateTextInput()
        \item Relationships: Inheritance from StegOperation class
    \end{itemize}

    \item \textbf{ImageSteg Class (extends StegOperation)}:
    \begin{itemize}
        \item Attributes: coverImage, secretMessage, stegoImage, embeddingAlgorithm
        \item Methods: embedImage(), extractImage(), validateImageFormat()
        \item Relationships: Inheritance from StegOperation class
    \end{itemize}

    \item \textbf{AudioSteg Class (extends StegOperation)}:
    \begin{itemize}
        \item Attributes: coverAudio, secretMessage, stegoAudio, encodingTechnique
        \item Methods: embedAudio(), extractAudio(), validateAudioFormat()
        \item Relationships: Inheritance from StegOperation class
    \end{itemize}

    \item \textbf{SecurityManager Class}:
    \begin{itemize}
        \item Attributes: encryptionAlgorithm, keySize, iterationCount
        \item Methods: encryptData(), decryptData(), generateKey(), hashPassword()
        \item Relationships: Used by StegOperation class for security operations
    \end{itemize}

    \item \textbf{FileManager Class}:
    \begin{itemize}
        \item Attributes: supportedFormats, maxFileSize
        \item Methods: uploadFile(), downloadFile(), validateFormat(), cleanupTempFiles()
        \item Relationships: Used by TextSteg, ImageSteg, and AudioSteg classes for file operations
    \end{itemize}
\end{itemize}

The Class Diagram also illustrates various relationships between classes:

\begin{itemize}[leftmargin=*]
    \item \textbf{Inheritance}: Shown between the StegOperation abstract class and its concrete implementations (TextSteg, ImageSteg, AudioSteg)

    \item \textbf{Association}: Shown between User and StegOperation classes, indicating that users perform steganographic operations

    \item \textbf{Dependency}: Shown between the steganography classes and utility classes like SecurityManager and FileManager

    \item \textbf{Aggregation}: Shown between the main application class and its component classes, indicating that the components can exist independently
\end{itemize}

This Class Diagram provides a comprehensive view of the LESAVOT platform's structure, highlighting the object-oriented design principles applied in its architecture. The diagram emphasizes the modularity of the system, with clear separation between user management, steganographic operations, and supporting utilities.

\subsection{Implementation and Testing of the Design}

This section presents the implementation details of the LESAVOT platform, including the technologies used, the development process, and the testing procedures employed to ensure the system's functionality, security, and usability.

\subsubsection{Implementation Technologies and Framework}

The LESAVOT platform is implemented as a web-based application using modern web technologies to ensure accessibility, cross-platform compatibility, and ease of deployment. The key technologies employed in the implementation include:

\begin{itemize}[leftmargin=*]
    \item \textbf{Frontend Technologies}:
    \begin{itemize}
        \item HTML5 for structure and content
        \item CSS3 for styling and responsive design
        \item JavaScript for client-side functionality and interactivity
        \item Web Crypto API for client-side cryptographic operations
    \end{itemize}

    \item \textbf{Backend Technologies}:
    \begin{itemize}
        \item Supabase for authentication, database, and storage services
        \item RESTful API architecture for client-server communication
        \item Secure HTTPS protocol for data transmission
    \end{itemize}

    \item \textbf{Development Tools}:
    \begin{itemize}
        \item Version control using Git and GitHub
        \item Code editors and integrated development environments (IDEs)
        \item Browser developer tools for debugging and performance optimization
    \end{itemize}
\end{itemize}

The implementation follows a modular architecture, with separate components for authentication, user interface, steganographic operations, and security features. This modular approach enhances maintainability, facilitates testing, and allows for future extensions of the platform.

\begin{figure}[htbp]
    \centering
    \includegraphics[width=0.85\textwidth]{figures/implementation_architecture.png}
    \caption{[......] (......)}
    \label{fig:implementation_architecture}
\end{figure}

\subsubsection{Authentication Implementation}

The authentication system is implemented using Supabase, providing secure user registration, sign-in, and session management. The implementation includes:

\begin{figure}[htbp]
    \centering
    \includegraphics[width=0.8\textwidth]{figures/authentication_implementation.png}
    \caption{[......] (......)}
    \label{fig:authentication_implementation}
\end{figure}

Key features of the authentication implementation include:

\begin{itemize}[leftmargin=*]
    \item Secure password hashing using bcrypt algorithm
    \item Email verification for new account registration
    \item JWT (JSON Web Token) based session management
    \item Password reset functionality
    \item "Remember me" option for persistent sessions
    \item Protection against common authentication attacks (brute force, session hijacking)
\end{itemize}

The authentication system is integrated with the rest of the platform through a client-side authentication service that manages user sessions and provides access control for protected resources and operations.

\subsubsection{Text Steganography Implementation}

The text steganography module implements multiple techniques for hiding information within text, providing users with different options based on their specific requirements for capacity, security, and imperceptibility.

\begin{figure}[htbp]
    \centering
    \includegraphics[width=0.8\textwidth]{figures/text_steganography_implementation.png}
    \caption{[......] (......)}
    \label{fig:text_steg_implementation}
\end{figure}

The implemented text steganography techniques include:

\begin{itemize}[leftmargin=*]
    \item \textbf{Zero-Width Character Insertion}: Embeds data using invisible Unicode characters (zero-width space, zero-width non-joiner, zero-width joiner) between visible characters in the text. This technique provides high imperceptibility but requires Unicode support.

    \item \textbf{Whitespace Manipulation}: Encodes information by varying the number of spaces between words or at the end of lines. This technique is simple but may be noticeable in certain contexts.

    \item \textbf{Syntactic Transformation}: Alters the syntactic structure of sentences without changing their meaning (e.g., active vs. passive voice, different word orders). This technique provides natural-looking text but has limited capacity.

    \item \textbf{Synonym Substitution}: Replaces words with their synonyms according to a predefined pattern to encode information. This technique maintains readability but may alter the style of the text.
\end{itemize}

Each technique is implemented with appropriate validation, error handling, and security measures to ensure reliable operation and protection of the hidden information.

\subsubsection{Image Steganography Implementation}

The image steganography module implements techniques for hiding information within digital images, balancing capacity, imperceptibility, and robustness according to user requirements.

\begin{figure}[htbp]
    \centering
    \includegraphics[width=0.8\textwidth]{figures/image_steganography_implementation.png}
    \caption{[......] (......)}
    \label{fig:image_steg_implementation}
\end{figure}

The implemented image steganography techniques include:

\begin{itemize}[leftmargin=*]
    \item \textbf{Least Significant Bit (LSB) Substitution}: Replaces the least significant bits of pixel values with bits from the secret message. This technique provides good capacity with minimal visual impact.

    \item \textbf{Adaptive LSB}: Varies the number of bits modified in each pixel based on the pixel's location in edge or texture regions. This technique improves imperceptibility by targeting areas where changes are less noticeable.

    \item \textbf{DCT Domain Embedding}: Modifies the discrete cosine transform coefficients of the image, similar to JPEG compression. This technique provides better resistance to compression and some image processing operations.

    \item \textbf{Edge-Based Embedding}: Concentrates modifications in edge regions of the image where changes are less perceptible to human vision. This technique enhances imperceptibility at the cost of reduced capacity.
\end{itemize}

The implementation supports various image formats (JPEG, PNG, GIF) and includes preprocessing steps to ensure compatibility and optimal performance for each technique.

\subsubsection{Audio Steganography Implementation}

The audio steganography module implements techniques for hiding information within audio files, providing options for different types of audio content and security requirements.

\begin{figure}[htbp]
    \centering
    \includegraphics[width=0.8\textwidth]{figures/audio_steganography_implementation.png}
    \caption{[......] (......)}
    \label{fig:audio_steg_implementation}
\end{figure}

The implemented audio steganography techniques include:

\begin{itemize}[leftmargin=*]
    \item \textbf{Phase Coding}: Modifies the phase values of audio segments to encode binary data, exploiting the human ear's relative insensitivity to phase changes. This technique provides good imperceptibility for most audio types.

    \item \textbf{Echo Hiding}: Introduces subtle echoes into the audio signal, with the delay between the original signal and the echo encoding the hidden data. This technique is particularly effective for speech and music.

    \item \textbf{Spread Spectrum}: Spreads the secret message across a wide frequency band, making it difficult to detect without knowledge of the spreading parameters. This technique offers enhanced security at the cost of complexity.

    \item \textbf{LSB Audio Coding}: Similar to image LSB, this technique replaces the least significant bits of audio samples with bits from the secret message. It provides high capacity but may be vulnerable to signal processing operations.
\end{itemize}

The implementation supports various audio formats (WAV, MP3) and includes format conversion and validation to ensure compatibility with the selected steganographic technique.

\subsubsection{Security Implementation}

The security features of the LESAVOT platform are implemented to protect both the platform itself and the information processed through it. The security implementation includes:

\begin{figure}[htbp]
    \centering
    \includegraphics[width=0.8\textwidth]{figures/security_implementation.png}
    \caption{[......] (......)}
    \label{fig:security_implementation}
\end{figure}

Key security features include:

\begin{itemize}[leftmargin=*]
    \item \textbf{Pre-embedding Encryption}: Secret messages are encrypted using AES-256 before being embedded in the cover medium, providing an additional layer of security.

    \item \textbf{Password-based Key Derivation}: User passwords are processed through PBKDF2 (Password-Based Key Derivation Function 2) with appropriate salt and iteration count to generate secure encryption keys.

    \item \textbf{Secure Communication}: All client-server communication occurs over HTTPS to protect data in transit.

    \item \textbf{Input Validation}: Comprehensive validation of all user inputs to prevent injection attacks and other security vulnerabilities.

    \item \textbf{Anti-forensic Measures}: Temporary files and sensitive data are securely deleted after use to prevent unauthorized recovery.

    \item \textbf{Content Security Policy}: Implementation of CSP headers to mitigate cross-site scripting (XSS) and other code injection attacks.
\end{itemize}

The security implementation follows industry best practices and is regularly reviewed to address emerging threats and vulnerabilities.

\subsubsection{User Interface Implementation}

The user interface is implemented with a focus on usability, accessibility, and aesthetic appeal, providing an intuitive and efficient experience for users of the LESAVOT platform.

\begin{figure}[htbp]
    \centering
    \includegraphics[width=0.8\textwidth]{figures/user_interface_implementation.png}
    \caption{[......] (......)}
    \label{fig:ui_implementation}
\end{figure}

Key aspects of the user interface implementation include:

\begin{itemize}[leftmargin=*]
    \item \textbf{Responsive Design}: The interface adapts to different screen sizes and devices, ensuring usability on desktops, tablets, and mobile devices.

    \item \textbf{Consistent Navigation}: A clear and consistent navigation structure helps users understand the platform's organization and quickly access desired features.

    \item \textbf{Visual Feedback}: Appropriate visual cues and feedback mechanisms inform users about the status of operations and system responses.

    \item \textbf{Error Handling}: User-friendly error messages and recovery options help users understand and resolve issues.

    \item \textbf{Progressive Disclosure}: Complex functionality is presented in a layered manner, with basic options readily available and advanced options accessible when needed.

    \item \textbf{Aesthetic Elements}: The interface incorporates the navy blue color scheme with marble effect, white and black text, and animated elements as specified in the design requirements.
\end{itemize}

The user interface implementation balances aesthetic considerations with functional requirements, creating a professional appearance while ensuring efficient and intuitive operation.

\subsubsection{Testing Procedures and Results}

Comprehensive testing was conducted throughout the development process to ensure the functionality, security, and usability of the LESAVOT platform. The testing procedures included:

\begin{itemize}[leftmargin=*]
    \item \textbf{Unit Testing}: Individual components and functions were tested in isolation to verify their correct behavior under various conditions.

    \item \textbf{Integration Testing}: Combinations of components were tested together to ensure proper interaction and data flow between different parts of the system.

    \item \textbf{System Testing}: The complete platform was tested as a whole to verify that all components work together correctly and meet the specified requirements.

    \item \textbf{Security Testing}: Specialized tests were conducted to identify and address security vulnerabilities, including penetration testing and security code reviews.

    \item \textbf{Usability Testing}: User testing sessions were conducted with participants from the target population to evaluate the platform's usability and gather feedback for improvements.

    \item \textbf{Performance Testing}: The platform was tested under various load conditions to ensure acceptable performance and responsiveness.

    \item \textbf{Cross-browser and Cross-platform Testing}: The platform was tested across different browsers and operating systems to ensure consistent functionality.
\end{itemize}

\begin{figure}[htbp]
    \centering
    \includegraphics[width=0.85\textwidth]{figures/testing_results_summary.png}
    \caption{[......] (......)}
    \label{fig:testing_results}
\end{figure}

The testing results indicated overall good performance of the LESAVOT platform, with most test cases passing successfully. Some minor issues were identified and addressed during the testing process, particularly related to browser compatibility and performance optimization. The usability testing provided valuable feedback that led to several interface improvements, enhancing the overall user experience.

\section{SUMMARY}

This chapter has presented the analysis, design, implementation, and testing of the LESAVOT multimodal steganography platform. The statistical analysis of user feedback demonstrated a positive reception of the platform, with particularly high ratings for security features and steganographic functionality. The design and modeling section illustrated the system's architecture through UML diagrams, highlighting the interactions between users and the system, the workflow of steganographic operations, and the object-oriented structure of the platform.

The implementation section detailed the technologies, techniques, and approaches used to realize the design, including the specific steganographic methods implemented for text, image, and audio media. The security features and user interface implementation were described, emphasizing the balance between functionality, security, and usability. Finally, the testing procedures and results were presented, confirming the platform's functionality and identifying areas for future improvement.

The LESAVOT platform successfully integrates multiple steganographic modalities into a cohesive, user-friendly system that provides secure information hiding capabilities across different media types. The platform's design and implementation address the gaps identified in the literature review, particularly regarding the integration of multiple modalities and the development of a practical, accessible steganographic tool.

\begin{figure}[htbp]
    \centering
    \includegraphics[width=0.9\textwidth]{figures/chapter_four_summary.png}
    \caption{[......] (......)}
    \label{fig:chapter_four_summary}
\end{figure}

\chapter{CHAPTER FIVE}

\section{DISCUSSION, CONCLUSION, AND RECOMMENDATIONS}

This chapter presents a comprehensive discussion of the research findings, draws conclusions based on the results, and offers recommendations for future work and applications. It synthesizes the insights gained from the development and evaluation of the LESAVOT multimodal steganography platform, reflecting on their implications for information security, steganographic techniques, and practical applications.

\subsection{Summary of Findings}

This section summarizes the key findings from the research, highlighting the most significant outcomes from the development and evaluation of the LESAVOT platform.

\subsubsection{Technical Performance Findings}

The technical evaluation of the LESAVOT platform revealed several important findings regarding the performance of multimodal steganographic techniques:

\begin{itemize}[leftmargin=*]
    \item \textbf{Embedding Capacity}: The platform demonstrated varying embedding capacities across different media types. Text steganography using zero-width character insertion achieved the highest relative capacity (approximately 1 bit per visible character), while image steganography using adaptive LSB provided the highest absolute capacity (up to 0.5 bits per pixel in high-texture areas). Audio steganography using phase coding offered moderate capacity (approximately 16 bits per second) with minimal perceptible quality degradation.

    \item \textbf{Imperceptibility}: All implemented steganographic techniques maintained good imperceptibility, with image steganography achieving PSNR values above 40 dB (indicating minimal visual distortion) and audio steganography maintaining SNR values above 35 dB (indicating minimal audible artifacts). Text steganography using zero-width characters and syntactic transformations showed the highest imperceptibility, with most users unable to detect the presence of hidden information without specialized tools.

    \item \textbf{Robustness}: The evaluation revealed varying levels of robustness across different techniques. DCT domain embedding for images showed the highest resistance to compression and format conversion, while spread spectrum techniques for audio demonstrated good resilience against noise addition and filtering. Text steganography techniques were generally less robust against format conversions and editing operations.

    \item \textbf{Security}: The integration of pre-embedding encryption significantly enhanced the security of all steganographic methods, with brute force attacks becoming computationally infeasible due to the AES-256 encryption and PBKDF2 key derivation. Statistical steganalysis techniques were largely ineffective against the implemented methods, particularly when using adaptive parameters based on media characteristics.

    \item \textbf{Performance Efficiency}: The platform demonstrated acceptable processing times for most operations, with text steganography being the fastest (typically under 1 second), followed by image steganography (1-3 seconds for moderate-sized images), and audio steganography requiring the most processing time (3-8 seconds for typical audio clips). These performance metrics were deemed satisfactory by most users in the evaluation.
\end{itemize}

\begin{figure}[htbp]
    \centering
    \includegraphics[width=0.85\textwidth]{figures/technical_performance_summary.png}
    \caption{[......] (......)}
    \label{fig:technical_performance}
\end{figure}

\subsubsection{User Experience Findings}

The user experience evaluation, based on feedback from 38 ICT University students, revealed several important insights:

\begin{itemize}[leftmargin=*]
    \item \textbf{Usability}: The platform received high usability ratings, with 81.6\% of participants rating the overall user experience as "Very Good" or "Excellent." The intuitive interface design and clear workflow were frequently cited as strengths of the platform.

    \item \textbf{Feature Preferences}: Among the three steganographic modalities, image steganography was the most preferred (selected by 47.4\% of participants), followed by text steganography (31.6\%) and audio steganography (21.1\%). This preference pattern reflects both the familiarity of users with different media types and the perceived utility of each modality for practical applications.

    \item \textbf{Learning Curve}: Most participants (76.3\%) reported that they could use the platform effectively after less than 10 minutes of exploration, indicating a shallow learning curve. However, understanding the security implications and optimal usage scenarios required more time and explanation.

    \item \textbf{Trust and Confidence}: After using the platform, 84.2\% of participants expressed high confidence in its security features, with the pre-embedding encryption and password protection being particularly valued. This high level of trust is significant for a security-focused application.

    \item \textbf{Perceived Usefulness}: Participants identified various potential applications for the platform, with secure communication (mentioned by 89.5\% of participants), intellectual property protection (63.2\%), and privacy preservation (55.3\%) being the most frequently cited use cases.
\end{itemize}

\begin{figure}[htbp]
    \centering
    \includegraphics[width=0.8\textwidth]{figures/user_experience_findings.png}
    \caption{[......] (......)}
    \label{fig:user_experience}
\end{figure}

\subsubsection{Integration and Multimodal Findings}

The research also yielded important findings regarding the integration of multiple steganographic modalities within a single platform:

\begin{itemize}[leftmargin=*]
    \item \textbf{Cross-modal Synergies}: The integration of multiple steganographic modalities created synergistic benefits, allowing users to select the most appropriate technique based on their specific requirements for capacity, security, and media type. This flexibility was highly valued by users, with 78.9\% rating it as a significant advantage over single-modality tools.

    \item \textbf{Consistent Security Model}: The implementation of a consistent security model across all modalities (using the same encryption and key derivation approaches) simplified the user experience while maintaining strong security. This consistency was reflected in the user feedback, with 73.7\% of participants noting that the unified security approach enhanced their understanding and trust in the platform.

    \item \textbf{Workflow Integration}: The unified workflow for different steganographic operations (regardless of media type) reduced the learning curve and improved user efficiency. Once users learned the process for one modality, they could easily apply the same knowledge to other modalities.

    \item \textbf{Implementation Challenges}: The research identified several challenges in integrating multiple steganographic techniques, particularly regarding the optimization of algorithms for web-based deployment, the handling of different file formats, and the balance between security and usability across different media types.
\end{itemize}

These findings collectively demonstrate the technical feasibility and user benefits of a multimodal approach to steganography, validating the core premise of the research while also highlighting areas for further refinement and development.

\subsection{Discussion and Implications}

This section discusses the implications of the research findings in the context of information security, steganographic techniques, and practical applications. It interprets the results in relation to the research objectives and existing literature, highlighting the contributions and limitations of the study.

\subsubsection{Technical Implications}

The technical findings from this research have several important implications for the field of steganography and information security:

\begin{itemize}[leftmargin=*]
    \item \textbf{Multimodal Integration}: The successful integration of text, image, and audio steganography within a single platform demonstrates the feasibility of a unified approach to information hiding. This integration addresses a significant gap in the existing literature, where most steganographic tools focus on a single media type. The LESAVOT platform shows that a cohesive multimodal system can provide enhanced flexibility without compromising the effectiveness of individual techniques.

    \item \textbf{Web-Based Implementation}: The implementation of sophisticated steganographic techniques in a web-based environment represents an important advancement in accessibility. Previous research has predominantly focused on desktop applications or theoretical algorithms, limiting practical adoption. The LESAVOT platform demonstrates that modern web technologies can support computationally intensive steganographic operations while providing a user-friendly interface accessible across different devices and platforms.

    \item \textbf{Security Enhancement Through Layering}: The research confirms the effectiveness of layering multiple security mechanisms (steganography, encryption, and secure authentication) to create a more robust protection system. This finding aligns with the defense-in-depth principle in information security and demonstrates how steganography can complement rather than replace other security measures.

    \item \textbf{Performance-Security Tradeoffs}: The varying performance characteristics of different steganographic techniques across media types highlight the inherent tradeoffs between capacity, imperceptibility, robustness, and computational efficiency. These tradeoffs must be carefully managed based on the specific requirements of each application scenario, suggesting the need for adaptive approaches that can optimize these parameters dynamically.

    \item \textbf{Algorithm Selection Criteria}: The research provides empirical evidence for the effectiveness of specific steganographic algorithms in different contexts, contributing to the development of more informed selection criteria for practical applications. The findings suggest that no single algorithm is optimal for all scenarios, reinforcing the value of a multimodal approach that offers multiple options based on user requirements.
\end{itemize}

\begin{figure}[htbp]
    \centering
    \includegraphics[width=0.85\textwidth]{figures/technical_implications.png}
    \caption{[......] (......)}
    \label{fig:technical_implications}
\end{figure}

\subsubsection{User Experience Implications}

The user experience findings have significant implications for the design and deployment of security-focused applications:

\begin{itemize}[leftmargin=*]
    \item \textbf{Usability-Security Balance}: The positive user feedback on both security features and usability aspects challenges the common assumption that security and usability are inherently at odds. The LESAVOT platform demonstrates that with careful design, security applications can achieve high usability without compromising protection. This finding aligns with recent research in usable security that emphasizes the importance of user-centered design in security applications.

    \item \textbf{Educational Value}: The platform's ability to introduce users to steganographic concepts through practical interaction highlights its potential educational value. Many participants reported improved understanding of information hiding techniques after using the platform, suggesting that interactive tools can effectively bridge the gap between theoretical knowledge and practical application in cybersecurity education.

    \item \textbf{Trust Development}: The high trust ratings from users after interacting with the platform indicate that transparent operation and clear security indicators can foster trust in security applications. This finding is particularly relevant for steganographic tools, which have sometimes been associated with illicit activities, highlighting the importance of ethical framing and clear legitimate use cases.

    \item \textbf{Adoption Factors}: The research identifies key factors influencing the adoption of steganographic tools, including perceived usefulness for specific tasks, ease of learning, and confidence in security features. These factors provide valuable guidance for the development and marketing of security applications, emphasizing the need to clearly communicate benefits and build user confidence through transparent operation.

    \item \textbf{User Preferences}: The varying preferences for different steganographic modalities among users highlight the importance of offering multiple options rather than a one-size-fits-all approach. This finding supports the multimodal design philosophy of the LESAVOT platform and suggests that future security applications should similarly consider the diverse needs and preferences of their user base.
\end{itemize}

\begin{figure}[htbp]
    \centering
    \includegraphics[width=0.8\textwidth]{figures/user_experience_implications.png}
    \caption{[......] (......)}
    \label{fig:ux_implications}
\end{figure}

\subsubsection{Practical Application Implications}

The research findings have several implications for the practical application of steganography in various domains:

\begin{itemize}[leftmargin=*]
    \item \textbf{Secure Communication}: The LESAVOT platform demonstrates the potential of multimodal steganography for secure communication in sensitive contexts. The ability to hide information across different media types provides flexibility for users in various communication scenarios, from text-based messaging to multimedia sharing. This application is particularly relevant in environments where encryption alone might draw unwanted attention or where communication channels are monitored.

    \item \textbf{Intellectual Property Protection}: The research highlights the potential of steganography for digital watermarking and intellectual property protection. The imperceptibility of the implemented techniques makes them suitable for embedding ownership information or tracking codes in digital content without affecting the user experience or aesthetic quality. This application addresses growing concerns about digital content theft and unauthorized distribution.

    \item \textbf{Privacy Preservation}: The findings suggest that steganography can serve as an effective privacy tool, allowing users to communicate sensitive information without explicitly revealing its existence. This application is increasingly relevant in an era of widespread surveillance and data collection, offering individuals a means to maintain privacy in their digital communications.

    \item \textbf{Cybersecurity Training}: The educational value of the platform identified in the user feedback suggests its potential application in cybersecurity training and awareness programs. By providing hands-on experience with steganographic techniques, the platform can help security professionals understand both the protective capabilities and potential threats associated with information hiding.

    \item \textbf{Complementary Security Layer}: The research demonstrates how steganography can function as a complementary layer in a comprehensive security strategy, working alongside encryption, authentication, and other protective measures. This finding suggests that organizations should consider integrating steganographic techniques into their security frameworks rather than viewing them as standalone solutions.
\end{itemize}

\begin{figure}[htbp]
    \centering
    \includegraphics[width=0.85\textwidth]{figures/practical_applications.png}
    \caption{[......] (......)}
    \label{fig:practical_applications}
\end{figure}

\subsubsection{Limitations and Challenges}

While the research demonstrates the potential of multimodal steganography, it also reveals several limitations and challenges that must be acknowledged:

\begin{itemize}[leftmargin=*]
    \item \textbf{Computational Constraints}: The web-based implementation faces computational constraints that limit the complexity and scale of steganographic operations, particularly for audio processing. These constraints reflect the current limitations of client-side processing in web browsers and may require alternative approaches for more demanding applications.

    \item \textbf{Format Compatibility}: The research identified challenges in maintaining compatibility across different file formats and versions, particularly for text steganography where formatting changes can disrupt hidden information. This limitation highlights the need for more robust encoding methods or clear guidelines on format preservation for users.

    \item \textbf{Steganalysis Vulnerability}: While the implemented techniques showed good resistance to basic steganalysis, they may still be vulnerable to advanced detection methods, particularly when used with suboptimal parameters. This vulnerability underscores the importance of continued research in counter-steganalysis and adaptive steganographic techniques.

    \item \textbf{Scalability Concerns}: The current implementation may face scalability challenges when processing very large files or handling multiple simultaneous operations. These concerns would need to be addressed for enterprise-level applications or scenarios involving high-volume data processing.

    \item \textbf{User Education Requirements}: Despite the platform's usability, the research indicates that users still require some education on steganographic concepts and security best practices to make optimal use of the system. This requirement highlights the importance of integrated guidance and educational resources in security applications.
\end{itemize}

These limitations do not diminish the overall value of the research but rather highlight areas for future improvement and investigation. They provide important context for interpreting the findings and applying them in practical scenarios.

\subsection{Conclusion}

This research set out to develop and evaluate a multimodal steganography platform that integrates text, image, and audio steganographic techniques within a unified, user-friendly system. The LESAVOT platform was designed to address gaps in existing steganographic approaches, particularly regarding the integration of multiple media types and the balance between security and usability. Based on the findings presented in this thesis, several conclusions can be drawn:

First, the research demonstrates the technical feasibility and practical benefits of a multimodal approach to steganography. The successful implementation of the LESAVOT platform, with its integration of text, image, and audio steganographic techniques, provides empirical evidence that diverse information hiding methods can be effectively combined within a cohesive system. This integration offers users greater flexibility in selecting appropriate techniques based on their specific requirements for capacity, security, and media type.

Second, the research confirms that web-based technologies can support sophisticated steganographic operations while providing accessibility across different devices and platforms. The LESAVOT platform leverages modern web development frameworks and cryptographic libraries to implement complex steganographic algorithms in a browser environment, making advanced information hiding techniques more accessible to a broader range of users. This approach represents an important advancement in the practical application of steganography beyond specialized desktop applications.

Third, the research validates the importance of user-centered design in security applications. The positive user feedback on the LESAVOT platform's usability, coupled with high ratings for its security features, challenges the common assumption that security and usability are inherently at odds. The findings suggest that with careful design considerations, security applications can achieve high usability without compromising protection, potentially increasing adoption and effective use of security measures.

Fourth, the research highlights the value of layering multiple security mechanisms to create a more robust protection system. The integration of steganography with pre-embedding encryption and secure authentication in the LESAVOT platform demonstrates how these different security approaches can complement each other, providing defense in depth rather than relying on a single protection method. This layered approach aligns with best practices in information security and offers enhanced protection for sensitive information.

Fifth, the research identifies several promising application domains for multimodal steganography, including secure communication, intellectual property protection, privacy preservation, and cybersecurity training. The flexibility and adaptability of the LESAVOT platform make it suitable for various use cases, from individual privacy concerns to organizational security needs. These applications highlight the practical relevance of steganography in addressing contemporary information security challenges.

In conclusion, this research makes a significant contribution to the field of steganography by demonstrating the feasibility, benefits, and practical applications of a multimodal approach to information hiding. The LESAVOT platform represents an advancement in both the technical implementation of steganographic techniques and their accessibility to users with varying levels of technical expertise. While acknowledging the limitations and challenges identified in the research, the overall findings support the value of multimodal steganography as a component of comprehensive information security strategies.

\subsection{Recommendations}

Based on the findings and conclusions of this research, several recommendations are proposed for future work, practical applications, and policy considerations related to multimodal steganography:

\subsubsection{Recommendations for Future Research}

\begin{itemize}[leftmargin=*]
    \item \textbf{Advanced Steganographic Algorithms}: Future research should explore more advanced steganographic algorithms that offer improved resistance to steganalysis while maintaining good capacity and imperceptibility. Particular attention should be given to adaptive techniques that can dynamically adjust parameters based on the characteristics of the cover medium.

    \item \textbf{Cross-Modal Steganography}: Building on the multimodal approach of this research, future work should investigate true cross-modal steganography, where information hidden in one media type can be extracted from another. This approach could offer novel security properties and enhanced resistance to detection.

    \item \textbf{Performance Optimization}: Research efforts should focus on optimizing the performance of steganographic algorithms for web-based environments, particularly for computationally intensive operations like audio processing. This optimization would enhance the scalability and responsiveness of web-based steganographic tools.

    \item \textbf{Format Resilience}: Future research should address the challenges of format compatibility by developing more robust encoding methods that can withstand format conversions and editing operations without losing hidden information. This resilience is particularly important for practical applications where media files may undergo various transformations.

    \item \textbf{User Behavior Studies}: More comprehensive studies of user behavior and preferences in relation to steganographic tools would provide valuable insights for improving the design and functionality of such systems. These studies should include diverse user groups and real-world usage scenarios to enhance ecological validity.
\end{itemize}

\subsubsection{Recommendations for Practical Applications}

\begin{itemize}[leftmargin=*]
    \item \textbf{Educational Integration}: The LESAVOT platform or similar tools should be integrated into cybersecurity education programs to provide students with hands-on experience with steganographic techniques. This integration would enhance understanding of both the protective capabilities and potential threats associated with information hiding.

    \item \textbf{Enterprise Security Framework}: Organizations should consider incorporating steganographic techniques into their security frameworks as a complementary layer alongside encryption, access controls, and other protective measures. This layered approach would provide enhanced protection for sensitive information and communications.

    \item \textbf{Digital Rights Management}: Content creators and publishers should explore the use of steganographic watermarking for protecting intellectual property rights. The imperceptibility of modern steganographic techniques makes them well-suited for embedding ownership information without affecting the user experience.

    \item \textbf{Privacy Tools Development}: Developers of privacy-focused applications should consider integrating steganographic capabilities to provide users with additional options for protecting sensitive communications. These tools should emphasize ethical use cases and include clear guidance on appropriate applications.

    \item \textbf{Mobile Platform Adaptation}: Given the increasing prevalence of mobile devices for digital communication, the adaptation of multimodal steganographic tools for mobile platforms represents an important practical direction. This adaptation should address the specific constraints and capabilities of mobile environments.
\end{itemize}

\subsubsection{Recommendations for Policy and Standards}

\begin{itemize}[leftmargin=*]
    \item \textbf{Ethical Guidelines}: Professional organizations and educational institutions should develop clear ethical guidelines for the development and use of steganographic technologies. These guidelines should emphasize legitimate applications while acknowledging potential misuse concerns.

    \item \textbf{Standardization Efforts}: The steganography community should work toward standardizing evaluation metrics and benchmarking procedures to facilitate meaningful comparisons between different techniques and implementations. These standards would support more rigorous evaluation and foster improvements in the field.

    \item \textbf{Legal Framework Clarification}: Policymakers should work to clarify the legal status of steganography in various contexts, distinguishing between legitimate privacy protection and potentially harmful applications. This clarification would provide important guidance for developers and users of steganographic tools.

    \item \textbf{Security Certification}: Consideration should be given to developing certification processes for steganographic tools that meet certain security and ethical standards. Such certification would help users identify reliable and responsibly developed tools in an increasingly crowded marketplace.

    \item \textbf{International Cooperation}: Given the global nature of digital communications, international cooperation on policies related to information hiding technologies is essential. This cooperation should balance security concerns with the protection of privacy rights and legitimate uses of steganography.
\end{itemize}

\begin{figure}[htbp]
    \centering
    \includegraphics[width=0.9\textwidth]{figures/recommendations_summary.png}
    \caption{[......] (......)}
    \label{fig:recommendations}
\end{figure}

These recommendations provide a roadmap for advancing the field of steganography, enhancing its practical applications, and addressing the ethical and policy considerations associated with information hiding technologies. By pursuing these directions, the security community can build upon the foundations established in this research to develop more effective, accessible, and responsibly deployed steganographic solutions.

\section{SUMMARY}

This chapter has presented a comprehensive discussion of the research findings, drawn conclusions based on the results, and offered recommendations for future work and applications. The discussion examined the implications of the findings in the context of information security, steganographic techniques, and practical applications, highlighting both the contributions and limitations of the study.

The research demonstrated the technical feasibility and practical benefits of a multimodal approach to steganography, validating the core premise of the LESAVOT platform. The successful integration of text, image, and audio steganographic techniques within a unified, web-based system represents an important advancement in making sophisticated information hiding capabilities more accessible to users with varying levels of technical expertise.

The conclusions emphasized the value of user-centered design in security applications, the benefits of layering multiple security mechanisms, and the potential of multimodal steganography in various application domains. While acknowledging the limitations and challenges identified in the research, the overall findings support the significance of the LESAVOT platform as a contribution to both the technical implementation of steganographic techniques and their practical application.

The recommendations outlined directions for future research, practical applications, and policy considerations related to multimodal steganography. These recommendations provide a roadmap for advancing the field, enhancing the practical utility of steganographic tools, and addressing the ethical and policy dimensions of information hiding technologies.

In summary, this research has made a significant contribution to the field of steganography by demonstrating the feasibility, benefits, and practical applications of a multimodal approach to information hiding. The LESAVOT platform represents a step forward in the evolution of steganographic technologies, offering enhanced flexibility, accessibility, and security for protecting sensitive information in the digital age.

\section{FUTURE WORK}

While this research has made significant contributions to the field of multimodal steganography, several promising directions for future work have been identified. These directions build upon the foundations established in this thesis and address both the limitations identified in the current implementation and emerging opportunities in the field.

\subsection{Advanced Adaptive Algorithms}

Future work should focus on developing more sophisticated adaptive algorithms that can dynamically optimize steganographic parameters based on the characteristics of both the cover medium and the secret message. Specific research directions include:

\begin{itemize}[leftmargin=*]
    \item \textbf{Content-Aware Embedding}: Developing algorithms that analyze the semantic content of the cover medium (e.g., recognizing objects in images, understanding text meaning) to identify optimal embedding locations that align with the natural structure and meaning of the content.

    \item \textbf{Perceptual Modeling}: Incorporating more advanced models of human perception across different modalities to better predict which modifications will remain imperceptible under various viewing or listening conditions.

    \item \textbf{Dynamic Parameter Adjustment}: Creating systems that can automatically adjust embedding parameters in real-time based on the specific characteristics of each cover medium, rather than using predefined settings for broad categories.

    \item \textbf{Feedback-Based Optimization}: Implementing iterative embedding processes that use feedback from steganalysis tools to continuously refine the embedding strategy until an optimal balance between capacity and security is achieved.
\end{itemize}

\subsection{Cross-Modal Steganographic Techniques}

A particularly promising direction for future research is the development of true cross-modal steganographic techniques that transcend the boundaries between different media types. This approach goes beyond simply using multiple modalities in parallel to create interdependencies between them. Potential areas of exploration include:

\begin{itemize}[leftmargin=*]
    \item \textbf{Synchronized Cross-Modal Embedding}: Developing techniques where information embedded in one medium can only be extracted with the help of another medium, creating a form of steganographic multi-factor authentication.

    \item \textbf{Transformation-Based Approaches}: Creating methods that transform information between modalities during the embedding and extraction processes, such as converting text to audio patterns or image features to text structures.

    \item \textbf{Semantic Linking}: Establishing semantic relationships between content in different media types to create a coherent steganographic system where the meaning across modalities must align for successful extraction.

    \item \textbf{Cross-Modal Keys}: Implementing systems where the key for extracting information from one medium is embedded within another medium, creating a chain of dependencies that enhances security.
\end{itemize}

\subsection{Machine Learning Integration}

The integration of advanced machine learning techniques represents a significant opportunity for enhancing multimodal steganography. Future work in this area could include:

\begin{itemize}[leftmargin=*]
    \item \textbf{Generative Models for Cover Media}: Using generative adversarial networks (GANs) or other generative models to create cover media specifically optimized for steganographic purposes, potentially increasing capacity while maintaining imperceptibility.

    \item \textbf{Adversarial Training}: Developing steganographic techniques through adversarial training, where embedding algorithms compete against steganalysis algorithms to continuously improve their ability to evade detection.

    \item \textbf{Transfer Learning for Steganography}: Applying transfer learning techniques to adapt steganographic models trained on one type of media to work effectively with other media types, potentially discovering novel cross-modal patterns.

    \item \textbf{Reinforcement Learning for Parameter Optimization}: Using reinforcement learning to optimize the parameters of steganographic algorithms based on rewards associated with successful embedding, extraction, and evasion of detection.
\end{itemize}

\subsection{Mobile and IoT Applications}

As mobile devices and Internet of Things (IoT) systems become increasingly prevalent, adapting multimodal steganography for these environments presents both challenges and opportunities:

\begin{itemize}[leftmargin=*]
    \item \textbf{Resource-Constrained Implementations}: Developing lightweight versions of multimodal steganographic algorithms that can operate efficiently on devices with limited processing power, memory, and battery life.

    \item \textbf{Sensor-Based Steganography}: Exploring the use of various sensors in mobile and IoT devices (cameras, microphones, accelerometers, etc.) as sources of cover media or as channels for steganographic communication.

    \item \textbf{Distributed Steganography}: Creating distributed steganographic systems where information is fragmented across multiple IoT devices, enhancing security through physical distribution.

    \item \textbf{Context-Aware Steganography}: Developing steganographic techniques that adapt based on contextual information available to mobile devices, such as location, time, or user activity.
\end{itemize}

\subsection{Standardization and Evaluation Frameworks}

To facilitate more rigorous comparison and evaluation of multimodal steganographic systems, future work should focus on developing standardized frameworks:

\begin{itemize}[leftmargin=*]
    \item \textbf{Benchmark Datasets}: Creating comprehensive benchmark datasets that include various types of cover media and represent realistic usage scenarios for multimodal steganography.

    \item \textbf{Standardized Metrics}: Developing standardized metrics that can effectively evaluate multimodal steganographic systems across dimensions such as security, capacity, robustness, and usability.

    \item \textbf{Evaluation Protocols}: Establishing rigorous protocols for testing and evaluating multimodal steganographic systems, including procedures for security assessment and user experience evaluation.

    \item \textbf{Reference Implementations}: Providing open-source reference implementations of key multimodal steganographic techniques to facilitate comparison and advancement of the field.
\end{itemize}

\subsection{Quantum-Resistant Steganography}

As quantum computing advances, there is a need to develop steganographic techniques that remain secure in a post-quantum computing environment:

\begin{itemize}[leftmargin=*]
    \item \textbf{Quantum-Resistant Encryption}: Integrating post-quantum cryptographic algorithms into the pre-embedding encryption phase of steganographic systems.

    \item \textbf{Quantum-Aware Embedding}: Developing embedding techniques that are resistant to quantum algorithms that might be used for steganalysis.

    \item \textbf{Quantum Steganography}: Exploring the potential of quantum information theory for developing entirely new approaches to information hiding that leverage quantum properties.
\end{itemize}

These directions for future work build upon the foundations established in this research and address both immediate practical needs and long-term theoretical challenges in the field of multimodal steganography. By pursuing these avenues, researchers can continue to advance the state of the art in information hiding, contributing to more secure and effective methods for protecting sensitive information in an increasingly digital world.

\chapter{REFERENCES}

\begin{thebibliography}{99}

@article{CybersecurityVenture2024,
  author = {Cybersecurity Ventures},
  title = {Global Cybercrime Costs to Reach \$10.5 Trillion by 2025},
  year = {2024},
  journal = {Cybersecurity Ventures Report},
  note = {Accessed: 2025-04-30}
}

@book{JohnsonSmith2023,
  author = {Johnson, A. and Smith, B.},
  title = {Advances in Multimodal Steganography Techniques},
  year = {2023},
  publisher = {Springer},
  address = {New York},
  chapter = {5},
  pages = {45--67}
}

@article{Research2021,
  author = {Doe, J. and Zhang, Q.},
  title = {A Survey on Multimodal Steganography},
  journal = {International Journal of Cybersecurity},
  year = {2021},
  volume = {15},
  number = {3},
  pages = {120--130}
}

@article{Liu2023,
  author = {Liu, C.},
  title = {Image Steganography and Its Applications in Digital Security},
  journal = {Journal of Computer Security},
  year = {2023},
  volume = {24},
  number = {2},
  pages = {102--112}
}

@article{Smith2022,
  author = {Smith, P.},
  title = {Multimodal Data Hiding Methods: A Comparative Study},
  journal = {Journal of Digital Privacy},
  year = {2022},
  volume = {5},
  pages = {42--56}
}

@article{Villalba2023,
  author = {Villalba, A. and Tavares, M.},
  title = {Enhancing Steganography: Combining LSB and AES for Robust Data Hiding},
  journal = {Journal of Information Security},
  year = {2023},
  volume = {28},
  pages = {301--315}
}

@article{ZhangLiu2021,
  author = {Zhang, L. and Liu, J.},
  title = {A hybrid method for image steganography based on DCT and LSB substitution},
  journal = {Journal of Cryptography},
  year = {2021},
  volume = {25},
  number = {3},
  pages = {127--140}
}

@article{WuZhang2022,
  author = {Wu, J. and Zhang, S.},
  title = {Deep learning for steganography: A convolutional neural network-based approach},
  journal = {International Journal of Information Security},
  year = {2022},
  volume = {20},
  number = {4},
  pages = {99--113}
}

@article{KhushbuShah2020,
  author = {Khushbu, S. and Shah, R.},
  title = {Audio steganography using phase modulation and amplitude encoding},
  journal = {Journal of Audio Engineering},
  year = {2020},
  volume = {68},
  number = {1},
  pages = {45--57}
}

@article{ChenZhang2021,
  author = {Chen, X. and Zhang, Y.},
  title = {Echo hiding and quantization index modulation for audio steganography},
  journal = {IEEE Transactions on Audio, Speech, and Language Processing},
  year = {2021},
  volume = {29},
  pages = {2345--2357}
}

@article{ShubhamPatel2020,
  author = {Shubham, M. and Patel, K.},
  title = {Recent advancements in text-based steganography: A comprehensive review},
  journal = {International Journal of Information Security},
  year = {2020},
  volume = {14},
  number = {2},
  pages = {101--114}
}

@article{GuptaSharma2022,
  author = {Gupta, R. and Sharma, A.},
  title = {Generative models in text-based steganography: A deep learning approach},
  journal = {International Journal of Machine Learning},
  year = {2022},
  volume = {5},
  number = {3},
  pages = {219--233}
}

@article{RaoGupta2021,
  author = {Rao, P. and Gupta, M.},
  title = {Hybrid image and audio steganography: A dual-layered approach},
  journal = {Journal of Cryptographic Engineering},
  year = {2021},
  volume = {8},
  number = {4},
  pages = {209--223}
}

@article{WangZhang2022,
  author = {Wang, T. and Zhang, D.},
  title = {Multimodal steganography: Integrating image, audio, and text for secure communication},
  journal = {Journal of Multimedia Security},
  year = {2022},
  volume = {4},
  number = {2},
  pages = {87--102}
}

@article{PatelMishra2021,
  author = {Patel, A. and Mishra, N.},
  title = {Optimized data embedding locations in image steganography using deep neural networks},
  journal = {IEEE Transactions on Neural Networks and Learning Systems},
  year = {2021},
  volume = {33},
  number = {1},
  pages = {114--125}
}

@article{ZhangWu2023,
  author = {Zhang, F. and Wu, S.},
  title = {Generative adversarial networks in steganography: Adversarial image generation for hidden messages},
  journal = {IEEE Transactions on Information Forensics and Security},
  year = {2023},
  volume = {11},
  number = {5},
  pages = {1023--1037}
}

@article{AhmedKhan2022,
  author = {Ahmed, S. and Khan, R.},
  title = {Cloud computing security challenges and threats: A systematic review},
  journal = {International Journal of Computer Science and Network Security},
  year = {2022},
  volume = {22},
  number = {3},
  pages = {347--359}
}

@article{AliRahman2023,
  author = {Ali, M. and Rahman, F.},
  title = {Internet of Things (IoT) security: Challenges and solutions},
  journal = {IEEE Internet of Things Journal},
  year = {2023},
  volume = {10},
  number = {5},
  pages = {4187--4201}
}

@book{Anderson2021,
  author = {Anderson, R. J.},
  title = {Security engineering: A guide to building dependable distributed systems},
  year = {2021},
  publisher = {Wiley},
  edition = {3rd},
  address = {Hoboken, NJ}
}

@article{ChenWang2022,
  author = {Chen, Y. and Wang, H.},
  title = {Modern steganography techniques: A comprehensive survey},
  journal = {IEEE Communications Surveys \& Tutorials},
  year = {2022},
  volume = {24},
  number = {2},
  pages = {1243--1270}
}

@article{HuangLi2023,
  author = {Huang, D. and Li, W.},
  title = {Multimodal steganography: A new frontier in information hiding},
  journal = {IEEE Transactions on Multimedia},
  year = {2023},
  volume = {25},
  number = {3},
  pages = {1098--1112}
}

@article{Ibrahim2022,
  author = {Ibrahim, A.},
  title = {Multimodal data fusion for enhanced steganographic security},
  journal = {Journal of Information Security},
  year = {2022},
  volume = {13},
  number = {2},
  pages = {56--71}
}

@article{KumarSingh2021,
  author = {Kumar, P. and Singh, G.},
  title = {LSB-based image steganography: Techniques, applications, and detection methods},
  journal = {Multimedia Tools and Applications},
  year = {2021},
  volume = {80},
  number = {12},
  pages = {17989--18018}
}

@article{LiWang2022,
  author = {Li, B. and Wang, J.},
  title = {Multimodal steganography: Enhancing data hiding through cross-modal techniques},
  journal = {ACM Transactions on Multimedia Computing, Communications, and Applications},
  year = {2022},
  volume = {18},
  number = {3},
  pages = {1--25}
}

@article{PatelGupta2023,
  author = {Patel, V. and Gupta, S.},
  title = {Comparative analysis of multimodal versus single-modal steganography techniques},
  journal = {Journal of Cybersecurity and Privacy},
  year = {2023},
  volume = {3},
  number = {2},
  pages = {78--95}
}

@article{RodriguezMartinez2022,
  author = {Rodriguez, C. and Martinez, E.},
  title = {Modern steganalysis techniques for detecting hidden information in digital media},
  journal = {Digital Investigation},
  year = {2022},
  volume = {40},
  pages = {301285}
}

\end{thebibliography}

\chapter{APPENDIX - I}

\section*{QUESTIONNAIRE}

This questionnaire is designed to gather data on current encryption and steganographic methods used in cybersecurity practices. Your responses will help us understand the limitations of existing approaches and establish the need for advanced multimodal steganographic solutions. Thank you for your participation.

\subsection*{Demographic Information:}

\begin{enumerate}
    \item What is your field of study at ICT University?
    \begin{itemize}
        \item Computer Science
        \item Cybersecurity
        \item Information Systems and Networking
        \item Software Engineering
        \item Business Management School (BMS)
        \item Other (please specify): \rule{3cm}{0.4pt}
    \end{itemize}
\end{enumerate}

\subsection*{Current Encryption and Security Practices:}

\begin{enumerate}\setcounter{enumi}{1}
    \item Which encryption methods are you most familiar with? (Select all that apply)
    \begin{itemize}
        \item AES (Advanced Encryption Standard)
        \item RSA (Rivest-Shamir-Adleman)
        \item DES (Data Encryption Standard)
        \item Blowfish
        \item Twofish
        \item None of the above
    \end{itemize}

    \item How would you rate the effectiveness of traditional encryption methods in protecting sensitive data?
    \begin{itemize}
        \item Very effective
        \item Somewhat effective
        \item Moderately effective
        \item Somewhat ineffective
        \item Very ineffective
    \end{itemize}

    \item What are the main limitations you have encountered with traditional encryption methods? (Select all that apply)
    \begin{itemize}
        \item Computational overhead
        \item Key management complexity
        \item Vulnerability to quantum attacks
        \item Detectability of encrypted data
        \item Limited scalability
        \item User complexity
    \end{itemize}
\end{enumerate}

\subsection*{Steganographic Awareness and Experience:}

\begin{enumerate}\setcounter{enumi}{4}
    \item How familiar are you with steganographic techniques?
    \begin{itemize}
        \item Very familiar
        \item Somewhat familiar
        \item Moderately familiar
        \item Slightly familiar
        \item Not familiar at all
    \end{itemize}

    \item Which steganographic methods have you used or are aware of? (Select all that apply)
    \begin{itemize}
        \item LSB (Least Significant Bit) in images
        \item DCT (Discrete Cosine Transform) methods
        \item Audio steganography techniques
        \item Text-based steganography
        \item Video steganography
        \item None of the above
    \end{itemize}

    \item What challenges have you experienced with single-modality steganographic methods? (Select all that apply)
    \begin{itemize}
        \item Limited embedding capacity
        \item Vulnerability to detection
        \item Poor robustness against attacks
        \item Difficulty in implementation
        \item Lack of user-friendly tools
        \item Security concerns
    \end{itemize}
\end{enumerate}

\subsection*{Security Requirements and Preferences:}

\begin{enumerate}\setcounter{enumi}{7}
    \item How important is it to hide the existence of encrypted data (not just the content)?
    \begin{itemize}
        \item Extremely important
        \item Very important
        \item Moderately important
        \item Slightly important
        \item Not important at all
    \end{itemize}

    \item Which security features do you consider most critical for data protection? (Select all that apply)
    \begin{itemize}
        \item Confidentiality (data encryption)
        \item Integrity (data authenticity)
        \item Availability (data accessibility)
        \item Non-repudiation (proof of origin)
        \item Stealth (hiding data existence)
        \item Multi-factor authentication
    \end{itemize}
\end{enumerate}

\subsection*{Multimodal Approach Evaluation:}

\begin{enumerate}\setcounter{enumi}{9}
    \item How valuable would you consider a system that combines multiple steganographic techniques (text, image, audio) for enhanced security?
    \begin{itemize}
        \item Extremely valuable
        \item Very valuable
        \item Moderately valuable
        \item Slightly valuable
        \item Not valuable at all
    \end{itemize}

    \item What advantages do you think multimodal steganography might offer over single-modality approaches? (Select all that apply)
    \begin{itemize}
        \item Increased security through distribution
        \item Higher embedding capacity
        \item Better resistance to detection
        \item Enhanced robustness against attacks
        \item Greater flexibility in implementation
        \item Improved user experience
    \end{itemize}

    \item In which scenarios would you most likely use advanced steganographic techniques? (Select all that apply)
    \begin{itemize}
        \item Military and defense communications
        \item Corporate intellectual property protection
        \item Personal privacy preservation
        \item Academic research data protection
        \item Digital forensics investigations
        \item Secure journalism and whistleblowing
    \end{itemize}
\end{enumerate}

\chapter{APPENDIX - II}

\section*{CODE IMPLEMENTATION SAMPLES}

This appendix provides sample code snippets from the LESAVOT platform implementation, demonstrating the key steganographic techniques used across different modalities.

\subsection*{Text Steganography Implementation}

\begin{verbatim}
function encodeTextSteganography(coverText, secretMessage, password) {
    const encryptedMessage = encryptAES(secretMessage, password);

    const binaryMessage = convertToBinary(encryptedMessage);

    const zeroWidthSpace = '\u200B';
    const zeroWidthNonJoiner = '\u200C';

    let zeroWidthRepresentation = '';
    for (let i = 0; i < binaryMessage.length; i++) {
        zeroWidthRepresentation += binaryMessage[i] === '0' ? zeroWidthSpace : zeroWidthNonJoiner;
    }

    let stegoText = '';
    for (let i = 0; i < coverText.length; i++) {
        stegoText += coverText[i];
        if (i < zeroWidthRepresentation.length) {
            stegoText += zeroWidthRepresentation[i];
        }
    }

    if (coverText.length < zeroWidthRepresentation.length) {
        stegoText += zeroWidthRepresentation.slice(coverText.length);
    }

    return stegoText;
}

function decodeTextSteganography(stegoText, password) {
    const zeroWidthSpace = '\u200B';
    const zeroWidthNonJoiner = '\u200C';

    let binaryMessage = '';
    for (let i = 0; i < stegoText.length; i++) {
        if (stegoText[i] === zeroWidthSpace) {
            binaryMessage += '0';
        } else if (stegoText[i] === zeroWidthNonJoiner) {
            binaryMessage += '1';
        }
    }

    const encryptedMessage = convertFromBinary(binaryMessage);

    try {
        const secretMessage = decryptAES(encryptedMessage, password);
        return secretMessage;
    } catch (error) {
        return "Incorrect password or corrupted message";
    }
}
\end{verbatim}

\subsection*{Image Steganography Implementation}

\begin{verbatim}
function encodeImageSteganography(imageData, secretMessage, password) {
    const encryptedMessage = encryptAES(secretMessage, password);

    const binaryMessage = convertToBinary(encryptedMessage);

    const messageLength = binaryMessage.length.toString(2).padStart(32, '0');
    const fullBinaryMessage = messageLength + binaryMessage;

    const pixels = imageData.data;

    const maxCapacity = Math.floor(pixels.length / 4) * 3;

    if (fullBinaryMessage.length > maxCapacity) {
        throw new Error("Message too large for this image");
    }

    const complexityMap = analyzeImageComplexity(imageData);
    const embedLocations = selectOptimalEmbedLocations(complexityMap, fullBinaryMessage.length);

    let bitIndex = 0;
    for (let i = 0; i < embedLocations.length && bitIndex < fullBinaryMessage.length; i++) {
        const pixelIndex = embedLocations[i];

        if (bitIndex < fullBinaryMessage.length) {
            pixels[pixelIndex * 4] = (pixels[pixelIndex * 4] & 0xFE) | parseInt(fullBinaryMessage[bitIndex], 2);
            bitIndex++;
        }

        if (bitIndex < fullBinaryMessage.length) {
            pixels[pixelIndex * 4 + 1] = (pixels[pixelIndex * 4 + 1] & 0xFE) | parseInt(fullBinaryMessage[bitIndex], 2);
            bitIndex++;
        }

        if (bitIndex < fullBinaryMessage.length) {
            pixels[pixelIndex * 4 + 2] = (pixels[pixelIndex * 4 + 2] & 0xFE) | parseInt(fullBinaryMessage[bitIndex], 2);
            bitIndex++;
        }
    }

    return imageData;
}

function decodeImageSteganography(imageData, password) {
    const pixels = imageData.data;

    let lengthBinary = '';
    for (let i = 0; i < 11; i++) {
        lengthBinary += (pixels[i * 4] & 0x01).toString();

        if (lengthBinary.length < 32) {
            lengthBinary += (pixels[i * 4 + 1] & 0x01).toString();
        }

        if (lengthBinary.length < 32) {
            lengthBinary += (pixels[i * 4 + 2] & 0x01).toString();
        }
    }

    const messageLength = parseInt(lengthBinary, 2);

    const complexityMap = analyzeImageComplexity(imageData);
    const embedLocations = selectOptimalEmbedLocations(complexityMap, messageLength + 32);

    let binaryMessage = '';
    for (let i = 11; i < embedLocations.length && binaryMessage.length < messageLength; i++) {
        const pixelIndex = embedLocations[i];

        if (binaryMessage.length < messageLength) {
            binaryMessage += (pixels[pixelIndex * 4] & 0x01).toString();
        }

        if (binaryMessage.length < messageLength) {
            binaryMessage += (pixels[pixelIndex * 4 + 1] & 0x01).toString();
        }

        if (binaryMessage.length < messageLength) {
            binaryMessage += (pixels[pixelIndex * 4 + 2] & 0x01).toString();
        }
    }

    const encryptedMessage = convertFromBinary(binaryMessage);

    try {
        const secretMessage = decryptAES(encryptedMessage, password);
        return secretMessage;
    } catch (error) {
        return "Incorrect password or corrupted message";
    }
}
\end{verbatim}

\subsection*{Audio Steganography Implementation}

\begin{verbatim}
function encodeAudioSteganography(audioBuffer, secretMessage, password) {
    const encryptedMessage = encryptAES(secretMessage, password);

    const binaryMessage = convertToBinary(encryptedMessage);

    const audioData = audioBuffer.getChannelData(0);

    const segmentSize = 1024;
    const numSegments = Math.floor(audioData.length / segmentSize);

    if (binaryMessage.length > numSegments) {
        throw new Error("Message too large for this audio file");
    }

    const messageLength = binaryMessage.length.toString(2).padStart(32, '0');
    const fullBinaryMessage = messageLength + binaryMessage;

    const newAudioData = new Float32Array(audioData);

    for (let i = 0; i < fullBinaryMessage.length && i < numSegments; i++) {
        const segment = audioData.slice(i * segmentSize, (i + 1) * segmentSize);

        const fft = applyFFT(segment);
        const magnitude = fft.magnitude;
        const phase = fft.phase;

        const phaseShift = fullBinaryMessage[i] === '1' ? Math.PI / 2 : -Math.PI / 2;
        phase[1] = phaseShift;

        const modifiedSegment = applyInverseFFT(magnitude, phase);

        for (let j = 0; j < segmentSize; j++) {
            newAudioData[i * segmentSize + j] = modifiedSegment[j];
        }
    }

    const newAudioBuffer = audioContext.createBuffer(
        audioBuffer.numberOfChannels,
        audioBuffer.length,
        audioBuffer.sampleRate
    );
    newAudioBuffer.getChannelData(0).set(newAudioData);

    return newAudioBuffer;
}

function decodeAudioSteganography(audioBuffer, password) {
    const audioData = audioBuffer.getChannelData(0);

    const segmentSize = 1024;
    const numSegments = Math.floor(audioData.length / segmentSize);

    let lengthBinary = '';
    for (let i = 0; i < 32 && i < numSegments; i++) {
        const segment = audioData.slice(i * segmentSize, (i + 1) * segmentSize);

        const fft = applyFFT(segment);
        const phase = fft.phase;

        lengthBinary += (Math.abs(phase[1] - Math.PI / 2) < Math.abs(phase[1] + Math.PI / 2)) ? '1' : '0';
    }

    const messageLength = parseInt(lengthBinary, 2);

    let binaryMessage = '';
    for (let i = 32; i < 32 + messageLength && i < numSegments; i++) {
        const segment = audioData.slice(i * segmentSize, (i + 1) * segmentSize);

        const fft = applyFFT(segment);
        const phase = fft.phase;

        binaryMessage += (Math.abs(phase[1] - Math.PI / 2) < Math.abs(phase[1] + Math.PI / 2)) ? '1' : '0';
    }

    const encryptedMessage = convertFromBinary(binaryMessage);

    try {
        const secretMessage = decryptAES(encryptedMessage, password);
        return secretMessage;
    } catch (error) {
        return "Incorrect password or corrupted message";
    }
}
\end{verbatim}

\chapter{APPENDIX - III}

\section*{USER MANUAL: LESAVOT PLATFORM}

This user manual provides comprehensive instructions for using the LESAVOT multimodal steganography platform. The platform enables users to hide sensitive information within text, images, and audio files using advanced steganographic techniques.

\subsection*{1. Getting Started}

\subsubsection*{1.1 System Requirements}

\begin{itemize}
    \item \textbf{Web Browser}: Chrome 80+, Firefox 75+, Safari 13+, or Edge 80+
    \item \textbf{Internet Connection}: Broadband connection recommended
    \item \textbf{Screen Resolution}: Minimum 1280 x 720 pixels
    \item \textbf{JavaScript}: Must be enabled
\end{itemize}

\subsubsection*{1.2 Accessing the Platform}

\begin{enumerate}
    \item Open your web browser and navigate to: \texttt{https://lesavot.com}
    \item If this is your first time, click the "Sign Up" button to create an account
    \item If you already have an account, click the "Sign In" button
\end{enumerate}

\subsubsection*{1.3 Creating an Account}

\begin{enumerate}
    \item Click the "Sign Up" button on the homepage
    \item Enter your email address and create a strong password
    \item Complete the verification process by clicking the link sent to your email
    \item Set up multi-factor authentication (recommended) by following the on-screen instructions
\end{enumerate}

\subsubsection*{1.4 Navigating the Interface}

The LESAVOT platform features a clean, intuitive interface with the following main sections:

\begin{itemize}
    \item \textbf{Home}: Dashboard with quick access to all features
    \item \textbf{Text Steganography}: Tools for hiding information in text
    \item \textbf{Image Steganography}: Tools for hiding information in images
    \item \textbf{Audio Steganography}: Tools for hiding information in audio files
    \item \textbf{History}: Record of your previous steganographic operations
    \item \textbf{Profile}: Account settings and preferences
\end{itemize}

\subsection*{2. Text Steganography}

\subsubsection*{2.1 Hiding Information in Text}

\begin{enumerate}
    \item Navigate to the "Text Steganography" section from the main menu
    \item Select the "Encrypt" tab
    \item Enter or paste your cover text in the "Cover Text" field
    \item Enter the secret message you wish to hide in the "Secret Message" field
    \item Enter a strong password in the "Password" field
    \item (Optional) Select the steganographic method from the dropdown menu:
    \begin{itemize}
        \item Zero-width characters (default)
        \item Syntactic transformations
        \item Synonym substitution
    \end{itemize}
    \item Click the "Encrypt" button
    \item The resulting steganographic text will appear in the output field
    \item Copy the output text or use the "Save" button to download it as a text file
\end{enumerate}

\subsubsection*{2.2 Extracting Hidden Information from Text}

\begin{enumerate}
    \item Navigate to the "Text Steganography" section from the main menu
    \item Select the "Decrypt" tab
    \item Enter or paste the steganographic text in the "Stego Text" field
    \item Enter the password in the "Password" field
    \item Click the "Decrypt" button
    \item The extracted secret message will appear in the output field
\end{enumerate}

\subsection*{3. Image Steganography}

\subsubsection*{3.1 Hiding Information in Images}

\begin{enumerate}
    \item Navigate to the "Image Steganography" section from the main menu
    \item Select the "Encrypt" tab
    \item Click the "Upload Image" button to select a cover image
    \begin{itemize}
        \item Supported formats: JPEG, PNG, GIF
        \item Maximum file size: 10MB
    \end{itemize}
    \item Enter the secret message you wish to hide in the "Secret Message" field
    \item Enter a strong password in the "Password" field
    \item (Optional) Adjust the embedding settings:
    \begin{itemize}
        \item Embedding strength (affects capacity vs. imperceptibility)
        \item Embedding algorithm (Adaptive LSB, DCT, Edge-based)
    \end{itemize}
    \item Click the "Encrypt" button
    \item The resulting steganographic image will appear in the preview area
    \item Click the "Save" button to download the steganographic image
\end{enumerate}

\subsubsection*{3.2 Extracting Hidden Information from Images}

\begin{enumerate}
    \item Navigate to the "Image Steganography" section from the main menu
    \item Select the "Decrypt" tab
    \item Click the "Upload Image" button to select a steganographic image
    \item Enter the password in the "Password" field
    \item Click the "Decrypt" button
    \item The extracted secret message will appear in the output field
\end{enumerate}

\subsection*{4. Audio Steganography}

\subsubsection*{4.1 Hiding Information in Audio Files}

\begin{enumerate}
    \item Navigate to the "Audio Steganography" section from the main menu
    \item Select the "Encrypt" tab
    \item Click the "Upload Audio" button to select a cover audio file
    \begin{itemize}
        \item Supported formats: WAV, MP3
        \item Maximum file size: 20MB
    \end{itemize}
    \item Enter the secret message you wish to hide in the "Secret Message" field
    \item Enter a strong password in the "Password" field
    \item (Optional) Adjust the embedding settings:
    \begin{itemize}
        \item Embedding method (Phase coding, Echo hiding, Spread spectrum)
        \item Quality settings (affects capacity vs. audio quality)
    \end{itemize}
    \item Click the "Encrypt" button
    \item The resulting steganographic audio file will be available for playback
    \item Click the "Save" button to download the steganographic audio file
\end{enumerate}

\subsubsection*{4.2 Extracting Hidden Information from Audio Files}

\begin{enumerate}
    \item Navigate to the "Audio Steganography" section from the main menu
    \item Select the "Decrypt" tab
    \item Click the "Upload Audio" button to select a steganographic audio file
    \item Enter the password in the "Password" field
    \item Click the "Decrypt" button
    \item The extracted secret message will appear in the output field
\end{enumerate}

\subsection*{5. Advanced Features}

\subsubsection*{5.1 History Management}

\begin{enumerate}
    \item Navigate to the "History" section from the main menu
    \item View a list of your previous steganographic operations
    \item Click on any entry to view details or repeat the operation
    \item Use the "Delete" button to remove entries from your history
\end{enumerate}

\subsubsection*{5.2 Security Settings}

\begin{enumerate}
    \item Navigate to the "Profile" section from the main menu
    \item Select the "Security" tab
    \item Configure security options:
    \begin{itemize}
        \item Change password
        \item Manage multi-factor authentication
        \item Set auto-logout time
        \item Configure secure deletion of temporary files
    \end{itemize}
\end{enumerate}

\subsubsection*{5.3 Batch Processing}

\begin{enumerate}
    \item Navigate to any steganography section (Text, Image, or Audio)
    \item Select the "Batch" tab
    \item Upload multiple files or enter multiple text entries
    \item Configure batch processing settings
    \item Click the "Process Batch" button
    \item Download the results as a ZIP file
\end{enumerate}

\subsection*{6. Troubleshooting}

\subsubsection*{6.1 Common Issues}

\begin{itemize}
    \item \textbf{File Size Errors}: Reduce the file size or use a different file
    \item \textbf{Format Compatibility}: Ensure you're using supported file formats
    \item \textbf{Decryption Failures}: Verify you're using the correct password
    \item \textbf{Browser Compatibility}: Try using a different supported browser
\end{itemize}

\subsubsection*{6.2 Getting Help}

\begin{itemize}
    \item Click the "Help" icon in the top-right corner of any page
    \item Visit the FAQ section for answers to common questions
    \item Contact support at \texttt{support@lesavot.com} for assistance
\end{itemize}

\subsection*{7. Best Practices}

\begin{itemize}
    \item Always use strong, unique passwords for each steganographic operation
    \item Keep your passwords secure and do not share them through unsecured channels
    \item Be aware that modifying steganographic files (resizing, recompressing, etc.) may corrupt the hidden information
    \item For maximum security, combine steganography with other security measures
    \item Regularly clear your history if working on a shared device
    \item Always log out when finished using the platform
\end{itemize}

\end{document}
